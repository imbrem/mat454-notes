\documentclass{article}
\usepackage[utf8]{inputenc}

\title{MAT454 Notes}
\author{Jad Elkhaleq Ghalayini}
\date{March 11 2020}

\usepackage{amsmath}
\usepackage{amssymb}
\usepackage{amsthm}
\usepackage{mathtools}
\usepackage{enumitem}
\usepackage{graphicx}
\usepackage{cancel}
\usepackage{mathabx}

\usepackage[margin=1in]{geometry}

\newtheorem{theorem}{Theorem}
\newtheorem{lemma}{Lemma}
\newtheorem*{claim}{Claim}

\newcommand{\brac}[1]{\left(#1\right)}
\newcommand{\sbrac}[1]{\left[#1\right]}
\newcommand{\eval}[3]{\left.#3\right|_{#1}^{#2}}
\newcommand{\ip}[2]{\left\langle#1,#2\right\rangle}
\newcommand{\mb}[1]{\mathbf{#1}}
\newcommand{\mbb}[1]{\mathbb{#1}}
\newcommand{\mc}[1]{\mathcal{#1}}
\newcommand{\prt}[2]{{\frac{\partial {#1}}{\partial {#2}}}}
\def\ries{{\hat{\mbb{C}}}}
\newcommand{\reals}{\mbb{R}}

\newtheorem{definition}{Definition}
\newtheorem{proposition}{Proposition}

\DeclareMathOperator{\Res}{Res}
\DeclareMathOperator{\BigP}{P}
\DeclareMathOperator{\Aut}{Aut}
\DeclareMathOperator{\Arg}{arg}
\DeclareMathOperator{\Id}{id}
\newcommand{\Prj}[2]{\BigP^{#1}({#2})}

\begin{document}

\maketitle

\section*{Theorems of Montel and Picard}

\subsection*{Picard's Big Theorem}

Picard's big theorem says that in the neighborhood of an essential singularity a holomorphic function omits at most one complex value. That is,
\begin{theorem}[Picard's Big Theorem]
If \(z_0\) is an isolated essential singularity of a holomorphic function \(f(z)\), then \(f\) takes every complex value with one possible exception in any neighborhood \(\Omega\) of \(z_0\), i.e. \(\#(\mbb{C}\setminus f(\Omega)) \leq 1\).
\end{theorem}
So, is this \textit{really} the best possible statement of this kind: that is, is it really possible that a complex function can omit one possible value? Yes: for example, \(e^{1/z} \neq 0\), so it omits value \(0\) even though it has an essential singularity at the origin.

So, there's also Picard's Little Theorem:
\begin{theorem}[Picard's Little Theorem]
A non-constant entire function \(f\) omits at most a point, i.e. \(\#(\mbb{C}\setminus f(\mbb{C})) \leq 1\)
\end{theorem}
\begin{proof}
So why does Picard's Little Theorem follow from Picard's Big Theorem (of course, Picard proved the little theorem first)? Well, either \(\infty\) is a pole or it is an essential singularity.
\begin{itemize}

  \item If \(\infty\) is a pole, by the fundamental theorem of algebra we have that \(f\) is a polynomial (since there are no poles at any finite points). So in this case, it does take \textit{every} value.

  \item If \(\infty\) is an essential singularity, then we can apply Picard's Big Theorem to a neighborhood of \(\infty\).

\end{itemize}
\end{proof}
So, we're going to prove Picard's big theorem using the theory of normal families, and in fact we're going to deduce it as well as a closely related, very strong theorem by Montel, from a strange lemma. This is going to be a necessary and sufficient condition for normality, but we'll express it as a necessary and sufficient condition for \textit{failure} of normality:
\begin{lemma}[Zalcman's Lemma]
Let \(\mc{S}\) be a family of meromorphic functions on a domain \(\Omega\). \(f\) is \underline{not} normal in the chordal metric if and only if there is a convergent sequence \(\{a_n\} \to a_\infty \in \Omega\), a convergent sequence of positive numbers \(\{\rho_n\} \to 0\) and a sequence of functions \(\{f_n\} \subset \mc{S}\) such that the sequence
\[g_n(z) = f_n(a_n + \rho_n z)\]
converges uniformly to \(g\) in the chordal metric on compact subsets of \(\mbb{C}\) where \(g\) is nonconstant and meromorphic on all of \(\mbb{C}\). Moreover, in the case where \(\mc{S}\) is not normal, then we can choose the data above such that
\[\forall z \in \mbb{C}, g^\sharp(z) \leq g^\sharp(0) = 1\]
\end{lemma}
So, what's strange about this lemma? Of course, it's strange because normality is about the existence of convergent sequences: non-normal should be that the stuff doesn't converge, and yet here we're saying that we can give a criterion for non-normality in terms of a convergent sequence. So, why should that be? I mean, like, what about the case where \(\mc{S}\) \textit{were} normal and we did this kind of construction: why would we get something worse? Or \textit{would} we get something worse?
That's sort of a quandry, and the point is in the case that \(\mc{S}\) is normal, i.e. every sequence in \(\mc{S}\) contains a convergent subsequence, it's not that we get something worse, it's that we get something better: we can still perform the above construction, but \(g\) would be constant!

So, what's an example? Consider \(\mc{S} = \{f_n(z) = z^n\}\). Of course, this is uniformly convergent on compact subsets of the unit disc, and actually, is also uniformly convergent in the chordal metric on compact subsets of the complement of the closed disc. But, it's not uniformly convergent on compact subsets of a \textit{bigger} open disc, e.g. \(|z| < 2\). The issue is that it's not uniformly continuous on the unit circle, as the limit would not be continuous there. So let's look at this situation, and take \(a_n = 1\), \(\rho_n = \frac{1}{n}\). That means that
\[g(z) = \lim_{n \to \infty}g_n(z) = \lim_{n \to \infty}\left(1 + \frac{z}{n}\right)^n = e^z\]
which is obviously nonconstant and entire. We know this from a proof from first year calculus, coming from the fact that
\[n\log(1 + z/n) \to z \iff \frac{\log(1 + z/n)}{z/n} \to 1\]
So, how do you compute the spherical derivative of \(g\)? So, it's just from the formula,
\[g^\sharp(z) = \frac{2|g'(z)|}{1 + |g(z)|^2} = \frac{2|e^z|}{1 + |e^z|^2} \implies g^\sharp(0) = \frac{2}{2} = 1, \quad g^\sharp(z) \leq 1 \impliedby \forall t \in \reals^+, 1 + t^2 \leq 2t\]
So that's sort of a phenomenon.

\end{document}
