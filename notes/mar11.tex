\documentclass{article}
\usepackage[utf8]{inputenc}

\title{MAT454 Notes}
\author{Jad Elkhaleq Ghalayini}
\date{March 11 2020}

\usepackage{amsmath}
\usepackage{amssymb}
\usepackage{amsthm}
\usepackage{mathtools}
\usepackage{enumitem}
\usepackage{graphicx}
\usepackage{cancel}
\usepackage{mathabx}

\usepackage[margin=1in]{geometry}

\newtheorem{theorem}{Theorem}
\newtheorem{lemma}{Lemma}
\newtheorem*{claim}{Claim}

\newcommand{\brac}[1]{\left(#1\right)}
\newcommand{\sbrac}[1]{\left[#1\right]}
\newcommand{\eval}[3]{\left.#3\right|_{#1}^{#2}}
\newcommand{\ip}[2]{\left\langle#1,#2\right\rangle}
\newcommand{\mb}[1]{\mathbf{#1}}
\newcommand{\mbb}[1]{\mathbb{#1}}
\newcommand{\mc}[1]{\mathcal{#1}}
\newcommand{\prt}[2]{{\frac{\partial {#1}}{\partial {#2}}}}
\def\ries{{\hat{\mbb{C}}}}
\newcommand{\reals}{\mbb{R}}

\newtheorem{definition}{Definition}
\newtheorem{proposition}{Proposition}

\DeclareMathOperator{\Res}{Res}
\DeclareMathOperator{\BigP}{P}
\DeclareMathOperator{\Aut}{Aut}
\DeclareMathOperator{\Arg}{arg}
\DeclareMathOperator{\Id}{id}
\newcommand{\Prj}[2]{\BigP^{#1}({#2})}

\begin{document}

\maketitle

\section*{Theorems of Montel and Picard}

\subsection*{Picard's Big Theorem}

Picard's big theorem says that in the neighborhood of an essential singularity a holomorphic function omits at most one complex value. That is,
\begin{theorem}[Picard's Big Theorem]
If \(z_0\) is an isolated essential singularity of a holomorphic function \(f(z)\), then \(f\) takes every complex value with one possible exception in any neighborhood \(\Omega\) of \(z_0\), i.e. \(\#(\mbb{C}\setminus f(\Omega)) \leq 1\).
\end{theorem}
So, is this \textit{really} the best possible statement of this kind: that is, is it really possible that a complex function can omit one possible value? Yes: for example, \(e^{1/z} \neq 0\), so it omits value \(0\) even though it has an essential singularity at the origin.

So, there's also Picard's Little Theorem:
\begin{theorem}[Picard's Little Theorem]
A non-constant entire function \(f\) omits at most a point, i.e. \(\#(\mbb{C}\setminus f(\mbb{C})) \leq 1\)
\end{theorem}
\begin{proof}
So why does Picard's Little Theorem follow from Picard's Big Theorem (of course, Picard proved the little theorem first)? Well, either \(\infty\) is a pole or it is an essential singularity.
\begin{itemize}

  \item If \(\infty\) is a pole, by the fundamental theorem of algebra we have that \(f\) is a polynomial (since there are no poles at any finite points). So in this case, it does take \textit{every} value.

  \item If \(\infty\) is an essential singularity, then we can apply Picard's Big Theorem to a neighborhood of \(\infty\).

\end{itemize}
\end{proof}
So, we're going to prove Picard's big theorem using the theory of normal families, and in fact we're going to deduce it as well as a closely related, very strong theorem by Montel, from a strange lemma.l

\end{document}
