\documentclass{article}
\usepackage[utf8]{inputenc}

\title{MAT454 Notes}
\author{Jad Elkhaleq Ghalayini}
\date{March 13 2020}

\usepackage{amsmath}
\usepackage{amssymb}
\usepackage{amsthm}
\usepackage{mathtools}
\usepackage{enumitem}
\usepackage{graphicx}
\usepackage{cancel}
\usepackage{mathabx}

\usepackage[margin=1in]{geometry}

\newtheorem{theorem}{Theorem}
\newtheorem{lemma}{Lemma}
\newtheorem*{claim}{Claim}

\newcommand{\brac}[1]{\left(#1\right)}
\newcommand{\sbrac}[1]{\left[#1\right]}
\newcommand{\eval}[3]{\left.#3\right|_{#1}^{#2}}
\newcommand{\ip}[2]{\left\langle#1,#2\right\rangle}
\newcommand{\mb}[1]{\mathbf{#1}}
\newcommand{\mbb}[1]{\mathbb{#1}}
\newcommand{\mc}[1]{\mathcal{#1}}
\newcommand{\prt}[2]{{\frac{\partial {#1}}{\partial {#2}}}}
\def\ries{{\hat{\mbb{C}}}}
\newcommand{\reals}{\mbb{R}}

\newtheorem{definition}{Definition}
\newtheorem{proposition}{Proposition}

\DeclareMathOperator{\Res}{Res}
\DeclareMathOperator{\BigP}{P}
\DeclareMathOperator{\Aut}{Aut}
\DeclareMathOperator{\Arg}{arg}
\DeclareMathOperator{\Id}{id}
\newcommand{\Prj}[2]{\BigP^{#1}({#2})}

\begin{document}

\maketitle

\section*{Proof of Zalcman's Lemma}

Let \(\mc{S}\) be a family of meromorphic functions on a domain \(\Omega\) which is \underline{not normal} in the chordal metric. We wish to find \(a_n \to a_\infty \in \Omega\), \(\rho_n \to 0\) such that
\[g_n(z) = f_n(a_n + \rho_n z)\]
converges to a nonconstant meromorphic function \(g\) in the chordal metric on compact subsets of \(\mbb{C}\) such that
\[g^\sharp(z) \leq g^\sharp(0) = 1\]
So, not normal means that
\[S^\sharp = \{f^\sharp: f \in \mc{S}\}\]
is not locally bounded (as per the theorem by Marty). This means, in particular, that there is a sequence of points \(b_n \to b_\infty \in \Omega\), \(f_n \in \mc{S}\) such that \(f_n^\sharp(b_n) \to \infty\). Of course, we can assume \(b_\infty = 0\), and hence in particular that there issomedisc \(\{|z| \leq r\} \subset \Omega\).

Let
\[M_n = \max(r - |\zeta|)f_n^\sharp(\zeta)\]
This is a continous function on a compact set, and so it takes on its maximum at some point \(a_n\), at which we have that this is equal to
\[(r - |a_n|)f_n^\sharp(a_n)\]
In particular, \(a_n \to \infty\) since \(b_n \to 0\). So the sequence we're going to construct is
\[g_n(z) = f_n(a_n + z/f_n^\sharp(a_n))\]
As \(n\) goes to infinity, this function is defined on bigger and bigger compact sets, as
\[\left|a_n + \frac{z}{f_n^\sharp(a_n)}\right| \leq |a_n| + \frac{|z|}{|f_n^\sharp(a_n)|} \leq |a_n| + r - |a_n|\]
That means that this is defined on \(|z| \leq M_n\). Fix \(R \geq 0\). If \(|z| \leq R \leq M_n\), then by the chain rule for spherical derivatives
\[|g_n^\sharp(z)| \leq \frac{|f_n^\sharp(a_n + z/f_n^\sharp(a_n))|}{|f_n^\sharp(a_n)|} \leq \frac{\cancel{M_n}}{r - |a_n + z/f_n^\sharp(a_n)|}\frac{r - |a_n|}{\cancel{M_n}} \leq \frac{r - |a_n|}{r - |a_n| - |z|/f_n^\sharp(a_n)} = \frac{1}{1  - |z|/M_n} \to 1\]
as \(n \to \infty\). This tells us by Marty's theorem that \(\{g_n\}\) contains a convergent subsequence in the chordal metric. Of course, taking that convergent subsequence and re-labeliling, we can just assume it is given by \(\{g_n\}\). \(g\) is meromorphic by this ``lemma", non-constant since \(g^\sharp(0) = 1\) and the endnote shows \(g^\sharp(z) \leq 1\).

\section*{Montel's Theorem}

We've already seen a result of Montel, that was the consequence of the Arzela-Ascoli theorem characterising normality in terms of uniformly bounded. This is a ``much souped up" version of that theorem, which says that if you have a family \(\mc{S}\) of meromorphic functions on a domain \(\Omega\) of \(\mbb{C}\) which omits three distinct values (\textit{including} \(\infty\)) in \(\mbb{C}^*\)l is normal in the chordal metric.

We'll do this on Monday (too bad there's a virus...), and we're going to see how the Big Picard Theorem follows from this. So, Big Picard's Theorem says that if \(f\) is meromorphic in a punctured disk \(0 < |z - z_0| < \rho\) and omits three distinct values in \(\mbb{C}^*\) then it extends to be meromorphic in \(|z - z_0| < \rho\).

Of course, this doesn't look like Picard's theorem as we stated it before, but they're in fact very easily seen to be equivalent.

\end{document}
