\documentclass{article}
\usepackage[utf8]{inputenc}

\title{MAT454 Notes}
\author{Jad Elkhaleq Ghalayini}

\usepackage{amsmath}
\usepackage{amssymb}
\usepackage{amsthm}
\usepackage{mathtools}
\usepackage{enumitem}
\usepackage{graphicx}
\usepackage{cancel}
\usepackage{mathabx}
\usepackage{xcolor}

\usepackage[margin=1in]{geometry}

\usepackage{hyperref}
\hypersetup{
  colorlinks,
  linkcolor={red!50!black},
  citecolor={blue!50!black},
  urlcolor={blue!80!black}
}

\newtheorem{theorem}{Theorem}
\newtheorem{lemma}{Lemma}
\newtheorem*{claim}{Claim}

\newcommand{\brac}[1]{\left(#1\right)}
\newcommand{\sbrac}[1]{\left[#1\right]}
\newcommand{\eval}[3]{\left.#3\right|_{#1}^{#2}}
\newcommand{\ip}[2]{\left\langle#1,#2\right\rangle}
\newcommand{\mb}[1]{\mathbf{#1}}
\newcommand{\mbb}[1]{\mathbb{#1}}
\newcommand{\mc}[1]{\mathcal{#1}}
\newcommand{\prt}[2]{{\frac{\partial {#1}}{\partial {#2}}}}
\def\ries{{\hat{\mbb{C}}}}
\newcommand{\reals}{\mbb{R}}

\newtheorem{definition}{Definition}
\newtheorem{proposition}{Proposition}

\DeclareMathOperator{\Res}{Res}
\DeclareMathOperator{\BigP}{P}
\DeclareMathOperator{\Aut}{Aut}
\DeclareMathOperator{\Arg}{arg}
\DeclareMathOperator{\Id}{id}
\newcommand{\Prj}[2]{\BigP^{#1}({#2})}

\begin{document}

\maketitle

These notes are based off a combination of my class notes and handwritten notes generously provided by Professor Edward Bierstone. While I made my best efforts to ensure their contents are correct, there may be a variety of errors and inconsistencies, all of which are my own.

\tableofcontents

\section{Review of Basic Complex Analysis}


The main objects of study in this course are holomorphic functions.
\begin{definition}[Holomorphic function]
\(f(t)\) is called \textbf{holomorphic at \(z \in \mbb{C}\)} if
\[\lim_{h \to 0}\frac{f(z + h) - f(z)}{h}\]
exists, i.e. there is \(c \in \mbb{C}\) such that
\[f(z + h) = f(z) + c \cdot h + \varphi(h) \cdot h, \lim_{h \to 0}\varphi(h) = 0\]
\end{definition}
Now, from this perspective, this looks no different from the usual case of a differentiable function. But it is different, because the variables are complex, and hence we can write
\[z = x + iy, \qquad f(z) = u(z) + iv(z)\]
Hence, this function mapping \(z \mapsto f(z)\) is, from the real perspective, a function from
\(\reals^2 \to \reals^2\), taking
\[\begin{pmatrix} x \\ y \end{pmatrix} \mapsto \begin{pmatrix} u(x, y) \\ v(x, y) \end{pmatrix}\]
Naturally, in the above definition, we can also write \(a + ib\) and \(h = \xi + i\eta\). Hence the derivative \(h \mapsto c \cdot h\) can be written as
\[\begin{pmatrix} \xi \\ \eta \end{pmatrix} \mapsto
\begin{pmatrix} a & -b \\ b & a \end{pmatrix}
  \begin{pmatrix} \xi \\ \eta \end{pmatrix}
= \begin{pmatrix}\prt{f}{x} & \prt{f}{y}\end{pmatrix}
  \begin{pmatrix} \xi \\ \eta \end{pmatrix}\]
In other words, this says that
\[\prt{f}{x} + i\prt{f}{y} = 0
\iff \prt{u}{x} = \prt{v}{y} \land \prt{u}{y} = -\prt{v}{x}\]
These are what is called the Cauchy-Riemann equations. So the moral of the story is that holomorphic is \textit{not} the same as differentiable as a function of two real variables. It' the same as differentiable as a function of two real variables \textit{plus} satisfying the Cauchy-Riemann equations.

It's going to convenient throughout this course to think about derivatives in terms of differential forms. Let's suppose, to begin a bit more generally, that we're considering a complex-valued \textit{differentiable} (not necessarily holomorphic) function \(f(x, y)\).
\begin{definition}
The \textbf{differential} of \(f\) is given by
\[df = \prt{f}{x}dx + \prt{f}{x}dy\]
\end{definition}
But, we're thinking about \(x\) and \(y\) as parts of a complex number, with \(z = x + iy\) and \(\bar{z} = x - iy\). So we can solve for \(x\) and \(y\) in terms of \(z\) and \(\bar{z}\). We can also compute the differentials
\[dz = dx + idy, \qquad d\bar{z} = dx - idy\]
So we can solve for \(dz\) and \(d\bar{z}\) in terms of \(dz\) and \(d\bar{z}\), getting
\[dx = \frac{1}{2}(dz + d\bar{z}), \qquad dy = \frac{1}{2i}(dz - d\bar{z})\]
In particular, we can take \(df\) and rewrite it in terms of \(dz\) and \(d\bar{z}\) by substituting in these expressions. So if we do that we get
\[df = \frac{1}{2}\left(\prt{f}{x} - i\prt{f}{y}\right)dz + \frac{1}{2}\left(\prt{f}{x} + i\prt{f}{y}\right)d\bar{z}\]
So, if we would like to define partial derivatives with respect to \(z\) and \(\bar{z}\), how should we define them? Well... the coefficients above seem to be natural choices, giving
\[\prt{f}{z} = \frac{1}{2}\left(\prt{f}{x} - i\prt{f}{y}\right),
  \qquad \prt{f}{\bar{z}}
= \frac{1}{2}\left(\prt{f}{x} + i\prt{f}{y}\right)
\implies df = \prt{f}{z}dz + \prt{f}{\bar{z}}d\bar{z}\]
In terms of \textit{this} expression, what's a third way of writing the Cauchy-Riemann equations? It's simply
\[\prt{f}{\bar{z}} = 0\]
And of course, this basically captures your ``feeling" of what a holomorphic function should be: it's supposed to be a function of \(z\), and not \(\bar{z}\). Ok, so this is the basic definition of holomorphic.

We'll now say a few words about harmonic functions. Recall the following definition
\begin{definition}[Harmonic]
We say a real or complex valued function \(f(x, y)\) is \textbf{harmonic} if \(f\) is \(\mc{C}^2\) and
\[\prt{^2f}{x^2} + \prt{^2f}{y^2} \iff \prt{^2f}{z\partial\bar{z}} = 0\]
\end{definition}
The above is known as \textbf{Laplace's equation}. It's immediate from the definition that a complex valued function is harmonic if and only if its real and imaginary parts are harmonic, and furthemore that every holomorphic function is harmonic. In particular, then, the real and imaginary parts of a holomorphic function are harmonic. On the other hand, maybe a slightly less immediate thing is that every real-valued harmonic function is, not necessarily everywhere but at least \textit{locally}, the real part of a holomorphic function. Why?

Well let's look at Laplace's equation. We know that Laplace's equation is satisfied, which tells us that
\[\prt{}{\bar{z}}\left(\prt{g}{z}\right) = 0\]
So this of course tells us that \(\prt{g}{z}\) is holomorphic. And why does the result follow from this? Because every holomorphic function locally has a primitive which is holomorphic. Where does that come from? The fact that a closed form is locally exact, which is essentially saying it is a consequence of Cauchy's theorem. Another way of thinking about it, which is really also saying it is a consequence of Cauchy's theorem, is that \(\prt{g}{z}\) is given by a convergent power series and hence can be locally integrated into another convergent power series. So this is really ``one way or another by Cauchy's theorem".

The global result, on the other hand, does not necessarily follow, in brief, because we can ``loop around once". For example, \(\log|z|\) is a real-valued harmonic function in \(\mbb{C} \setminus \{0\}\), but it's not \textit{globally} the real part of a holomorphic function in \(\mbb{C} \setminus \{0\}\), because \(\log z\) has no single-valued branch here. This is a counterexample, but not on \(\mbb{C}\). Whether there are counterexamples in \(\mbb{C}\) is a very good question, and we'll deal with that when we get to Cauchy's theorem. It definitely is a topological question.

This is just a very brief recollection of the basic definitions of holomorphic and harmonic functions. I want to also recall, though maybe not all of you are familiar with this, the definitions of the various kinds of functions we're going to be working with as well as the spaces these functions are going to be defined on. In particular, everyone in a first-year course in complex variables has seen in some way the fact that its reasonable to say what you mean by ``holomorphic at \(\infty\)", and it can be useful to think about that. So, what \textit{do} you mean when you say that \(f(z)\) is holomorphic at \(\infty\)? Without introduing anything new, we can say that this means \(f(1/z)\) is holomorphic at \(0\). This is a very useful thing. We would like to make sense of this in a sort of well-structured way, and one does that by extending the complex plane to include the point at \(\infty\), or rather, to think of our functions not as on the complex plane, but on the extended complex plane including the point at \(\infty\), which is also called the Riemann sphere.

We have to say what the complex structure of that space is in a neighborhood of infinity, in such a way that captures this intuition, such that our holomorphic functions are holomorphic functions defined on open neighborhoods of the Riemann sphere.
Of course, complex-valued functions which are holomorphic on the \textit{whole} Riemann sphere are rather uninteresting, considering they are all constant by Liouville's principle. If we're allowed to consider holomorphic functions on the Riemann sphere \textit{with values on the Riemann sphere}, however, then we're back in interesting territory.

\begin{definition}[Stereographic Projection]
Consider the unit sphere \(S^2 = \{x^2 + y^2 + z^2 = 1\}\), and identify \(\reals^2\) with \(\mbb{C}\) by the isomorphism \((x, y) \mapsto z = x + iy\). Define the north pole \((0, 0, 1)\). We can define the \textbf{stereographic projection from the north pole} from \(\pi: S^2 \setminus N \to \mbb{C}\) to map a point \(s \in S \setminus N\) to the intersection of the line between \(s = (x, y, t)\) and \(N\) and the \(xy\) plane. Because the points \(s, N, (x/(1 - t), y/(1 - t), 0)\) must be colinear, we can define
\[\pi(x, y, t) = \frac{x + iy}{1 - t}\]
This is a homeomorphism from \(S^2 \setminus N\) to \(\mbb{C}\).
\end{definition}
A quick question: is this a \textit{metric} isomorphism? \textbf{No}: points very close together on the sphere can map to points very far from each other in the plane. This, however, is going to be a very important point in this course: we will study the behaviour of holomorphic functions according to the two natural metrics on the sphere: the induced metric on \(\reals^3\) i.e. the \textbf{chordal metric}, equivalent to the \textbf{geodesic metric}.

So the above homeomorphism gives a complex structure to the unit sphere minus the north pole. If we wanted to, we could get a complex structure on the unit sphere minus the south pole \(S = (0, 0, -1)\) by taking the stereographic projection from there. But we don't want to do that, because the complex structure we'd get would be incompatible. Instead, we want to take the \textit{complex conjugate} of a stereographic projection from the south pole,
\[z' = \frac{x - iy}{1 + t}\]
So what was the point about compatibility? Well, what's the relationship between \(z\) and \(z'\)? We have
\[z \cdot z' = \frac{x^2 + y^2}{(1 - t)^2} = 1 \implies z' = \frac{1}{z}\]
This is a holomorphic function from \(\mbb{C} \setminus \{0\} \to \mbb{C} \setminus \{0\}\) with a holomorphic inverse. So the two complex structures defined on the sphere minus the north pole and the sphere minus the south pole are compatible. By a \textit{complex structure} on a set, we mean a homeomorphism between an open subset that set and an open subset of the complex plane. If you're familiar with the language of manifolds, each of these two mappings is a \textit{coordinate chart}. These are even better than manifolds, though, because the coordinate charts are not just differentiable or infinitely differentiable, but holomorphic, or even better, \textit{rational}.

\section{Elliptic Functions}

...


\section{The Conformal Mapping Problem}

Let \(f\) be a holomorphism, and assume \(f'(z_0) = 0\). Then \(f^{-1}\) exists in a neighborhood of \(f(z)\), and \(f\) is \textbf{conformal} at \(z_0\) (preserves angles and their orientations). A nonconstant holomorphic mapping \(f: \Omega \to \mbb{C}\) is \textbf{open}, if it is one to one, then \(f\) is a homeomorphism onto its image \(f(\Omega)\), and \(f^{-1}\) is a holomorphism.
\begin{definition}
A \textbf{conformal} or \textbf{biholomorphic} mapping \(f: \Omega \to \Omega'\) is a holomorphic mapping with aholomorphic inverse.
\end{definition}
We are now faced with the \textbf{conformal mapping problem}:
\begin{itemize}

  \item Given domains \(\Omega, \Omega' \subset \mbb{C}\), are they biholomorphic?

  \item If so, can we find all biholomorphisms?

\end{itemize}
We note that, for \(f, g: \Omega \to \Omega'\), \(f, g\) are biholomorphisms if and only if \(g^{-1} \circ f \in \Aut\Omega\), the group of biholomorphisms of \(\Omega\) with itself. Furthermore, \(f\) induces a conjugation map
\[\Aut\Omega \to \Aut\Omega', \quad S \mapsto f \circ S \circ f^{-1}\]
Now let's consider some examples, starting with the complex plane itself. We have that
\[\Aut\mbb{C} = \{\text{linear transformations} \ w = az + b, \quad a \neq 0\}\]
Suppose \(w = f(z) \in \Aut\mbb{C}\). At \(\infty\), \(f\) has either an essential signularity or a pole. But we can show we don't have an essential singularity. On the other hand, what about when \(f\) is a polynomial, say of degree \(n\), then that must mean it is not one-to-one, because
\(f(z) = w\) has \(n\) distinct roots for almost every value of \(w\), except at roots of the derivative \(w = f(z), f'(z) = 0\). So \(n = 1\).

What about the Riemann sphere, \(\Aut S^2\). What should this group of biholomorphisms look like? Wec consider fractional linear transformations
\[w = \frac{az + b}{cz + d}, \quad ad - bc \neq 0\]
These coefficient, of course, are not uniquely determined, being only determined up to a constant. The inverse of a fractional linear transformation is that given by the inverse matrix, namely
\[\frac{dz - b}{-cz + a}\]
(uniquely determined up to a constant, so we don't have to write the \(\frac{1}{ad - bc}\)).

\begin{lemma}
Suppose \(G\) is a subgroup of \(\Aut\Omega\) such that \(G\) is transitive on \(\Omega\) and for some \(z_0\) the subgroup of automorphisms which fix this point lies inside of \(G\). Then \(G = \Aut\Omega\).
\end{lemma}
\begin{proof}
Let \(S \in \Aut\Omega\) be arbitrary. We have to show that \(S \in G\). To do so, we note that since \(G\) acts transitively, we can take \(T \in G\) such that \(T(z_0) = S(z_0)\). There exists such a \(T\) because it acts transitively. Then of course we can write
\[S = T \circ (T^{-1} \circ S), \quad (T^{-1} \circ S)(z_0) = z_0 \implies T^{-1} \circ S \in G \implies S \in G\]
being the composition of elements of \(G\).
\end{proof}
So that's what we've shown here: the subgroup of automorphisms fixing \(\infty\) lies in this subgroup of fractional linear transformations, and the subgroup is transitive, and hence it composes the entire automorphism group \(\Aut S^2\).

Time for another example: the unit disc \(D\). So what would \(\Aut D\) look like? What would we like to show here? I guess we'd like to show that they're all fractional linear transformations, but which ones? This is sort of like what Schwarz's lemma tells you: if you have an automorphism of the disk that fixes just zero, then it should be a rotation. So in general, it's like a rotation times a factor
\[w = e^{i\theta}\frac{z - z_0}{1 - \bar{z_0}z}\]
So how do you check that this actually is an automorphism of \(D\)? First of all, we should check that the boundary goes to the boundary. We can check this by just checking three points that are particularly nice, say \(1, i, -i\). Once we know this, we just need to check that the inside goes to the inside, since being an automorphism of \(S^2 \to S^2\), it either takes the inside to the inside biholomorphically or takes the inside to the outside. So if we add the condition \(|z_0| < 1\), we get that they take the inside to the inside.

Now how do we show that these compose \textit{all} automorphisms. It's not actually going to be by the previous lemma, rather, we will use Schwarz's lemma. Suppose \(T \in \Aut D\), and consider
\[S(z) = e^{i\theta}\frac{z - z_0}{1 - \bar{z_0}z}, \qquad \text{where} \ z_0 = T(0), \quad \theta = \Arg T'(z_0)\]
We want to show that \(T = S\), by Schwarz's lemma. So what should we apply the lemma to? \(f = S \circ T^{-1}\) is an obvious choice (we could also try \(T \circ S^{-1}\), etc...). So what do we know? We have that:
\begin{itemize}

  \item \(f(0) = 0\)

  \item \(f(D) = D \implies [|z| < 1 \implies |f(z)| < 1]\)

\end{itemize}
So by Schwarz's lemma \(|f(z)| \leq z\) for all \(z \in D\). We can also apply Scharz's lemma to \(f^{-1}\) and we get \(|z| \leq |f(z)|\), which says that in modulus
\[|f(z)| = |z|\]
which, again by Schwarz's lemma, says \(f\) is a rotation \(e^{i\alpha}z\). Now, we're basically done here, as choosing \(z_0 = 0\), this lies in the subgroup \(G\) and hence \(S \in G\). But we want to go further and show \(\alpha = 0\) implying \(S = T\). To do so, we merely note that
\[S'(z_0) = T'(z_0) \implies \alpha = 0\]
So this is in fact what we really need for the Riemann mapping theorem. Let's finish this up: what's the other nice domain we should look at? The upper half plane!

So what's \(\Aut\mbb{H}^+\)? So first of all, the upper half-plane is actually biholomorphic to the unit disc \(D\), by, for example, the mapping
\[\frac{z - i}{z + i}\]
To see this, we note that the boundary goes to the boundary, by checking the three points \(0, 1, \infty\) (which go to \(-1, -i, 1\) respectively). Of course the upper half-plane goes to the interior of the disc since \(i \mapsto 0\). Hence, the automorphism groups of \(D\) and \(\mbb{H}^+\) are the same, as we can transport elements between the two by composing with a biholomorphism as above. Geometrically, however, this is not going to tell us the form of these holomorphisms. So what should \(\Aut\mbb{H}^+\) look like in terms of holomorphisms of the Riemann sphere? Well, it should be the one with real coefficients, as it should be the subgroup of \(\Aut S^2\) taking \(\mbb{H}^+\) to itself and hence the real line to itself. So I claim the best form is
\[w = \frac{az + b}{cz + d}, \quad a, b, c, d \in \mbb{R}\]
Since these are only determined up to a constant, for convenience we can say \(ad - bc = \pm 1\). This is the subgroup of \(\Aut S^2\) taking the real axis \(\mbb{R}\) to itself. But if it takes the upper half plane to itself, that says that
\[\Im\left(\frac{ai + b}{ci + d}\right) = \frac{ad - bc}{c^2 + d^2} > 0 \iff ad - bc > 0 \iff ad > bc\]
So we have a subgroup \(G\) of fractional linear transformations in the form above satisfying the given conditions, i.e. with
\[w = \frac{az + b}{cz + d}, \quad ad - bc = 1, a, b, c, d \in \mbb{R}\]
So we want to see that \(\Aut\mbb{H}^+ = G\). This time, we'll use the Lemma: \(G\) is a subgroup of \(\Aut\mbb{H}^+\), it's transitive on \(\Aut\mbb{H}^+\), as we can show that it can take \(i\) to any given element of \(\mbb{H}^+\), since
\[i \mapsto ai + b, b > 0, ... TODO\]
So we have to show that the subgroup of \(\Aut\mbb{H}^+\) which fixes some point lies inside of this. So which point do we want to use? \(i\). How do we show this?

\(G\) is the subgroup of all fractional linear transformations which take the upper half-plane to itself. So all we have to do is show that this subgroup consists of fractional linear transformations. Why does it consist of fractional linear transformations? It's enough to show that the subgroup of \(\Aut\mbb{H}^+\) which fixes \(i\) consists of fractional linear transformations. That's because this subgroup of \(\Aut\mbb{H}^+\) which fixes \(i\) is just obtained from the group of automorphisms of \(D\) which fix zero by conjugation with \(\frac{z - i}{z + i}\). Anthing in here is a composite of three fractional linear transformations, and so is a fractional linear transformation.

So this is meant to be an exercise which should essentially be recalling things. Next time we'll prove the Riemann mapping theorem using normal mappings and we'll at least need these.

\end{document}
