\documentclass{article}
\usepackage[margin=1in]{geometry}
\usepackage{mat454}

\usepackage[backend=biber]{biblatex}
\addbibresource{references.bib}

\usepackage{hyperref}
\hypersetup{
  colorlinks,
  linkcolor={red!50!black},
  citecolor={blue!50!black},
  urlcolor={blue!80!black}
}

\title{MAT454 Academic Offense Sheet}
\author{Jad Elkhaleq Ghalayini}

\begin{document}

\maketitle

A quick collection of useful facts, theorems, and definitions for complex analysis. May be incorrect, and is certainly incomplete. Use at your own risk!

\tableofcontents

\newpage

\section{Basic Definitions and Theorems}
For \(f = u + iv\) holomorphic, we have
\begin{equation}
  2\prt{f}{\bar{z}} = \prt{f}{x} + i\prt{f}{y} = 0
  \iff \prt{u}{x} = \prt{v}{y} \land \prt{u}{y} = -\prt{v}{x}
\end{equation}
\begin{definition}
The \textbf{differential} of \(f\) is given by
\begin{equation}
  df = \prt{f}{x}dx + \prt{f}{y}dy = \prt{f}{z}dz + \prt{f}{\bar{z}}d\bar{z}
\end{equation}
\end{definition}
\begin{align}
  dz = dx + idy, \qquad d\bar{z} = dx - idy \iff
  dx = \frac{1}{2}(dz + d\bar{z}), \qquad dy = \frac{1}{2i}(dz - d\bar{z}) \\
  \prt{f}{z} = \frac{1}{2}\left(\prt{f}{x} - i\prt{f}{y}\right),
    \qquad \prt{f}{\bar{z}}
    = \frac{1}{2}\left(\prt{f}{x} + i\prt{f}{y}\right)
    \implies df = \prt{f}{z}dz + \prt{f}{\bar{z}}d\bar{z}
\end{align}
\begin{definition}[Harmonic]
We say a real or complex valued function \(f(x, y)\) is \textbf{harmonic} if \(f\) is \(\mc{C}^2\) and
\begin{equation}
  \prt{^2f}{x^2} + \prt{^2f}{y^2} \iff \prt{^2f}{z\partial\bar{z}} = 0
\end{equation}
\end{definition}
\begin{proposition}
  Every real-valued harmonic function is, not necessarily everywhere but at least \textit{locally}, the real part of a holomorphic function.
\end{proposition}

\section{Useful Tools}

\section{Residues and Integrals}

\section{Elliptic Curves}

\end{document}
