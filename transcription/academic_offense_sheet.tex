\documentclass{article}
\usepackage[margin=1in]{geometry}
\usepackage{mat454}

\usepackage[backend=biber]{biblatex}
\addbibresource{references.bib}

\usepackage{hyperref}
\hypersetup{
  colorlinks,
  linkcolor={red!50!black},
  citecolor={blue!50!black},
  urlcolor={blue!80!black}
}

\title{MAT454 Academic Offense Sheet}
\author{Jad Elkhaleq Ghalayini}

\begin{document}

\maketitle

A quick collection of useful facts, theorems, and definitions for complex analysis. May be incorrect, and is certainly incomplete. Use at your own risk!

\tableofcontents

\newpage

\section{Basic Definitions and Theorems}
For \(f = u + iv\) holomorphic, we have
\begin{equation}
  2\prt{f}{\bar{z}} = \prt{f}{x} + i\prt{f}{y} = 0
  \iff \prt{u}{x} = \prt{v}{y} \land \prt{u}{y} = -\prt{v}{x}
\end{equation}
\begin{definition}
The \textbf{differential} of \(f\) is given by
\begin{equation}
  df = \prt{f}{x}dx + \prt{f}{y}dy = \prt{f}{z}dz + \prt{f}{\bar{z}}d\bar{z}
\end{equation}
\end{definition}
\begin{align}
  dz = dx + idy, \qquad d\bar{z} = dx - idy \iff
  dx = \frac{1}{2}(dz + d\bar{z}), \qquad dy = \frac{1}{2i}(dz - d\bar{z}) \\
  \prt{f}{z} = \frac{1}{2}\left(\prt{f}{x} - i\prt{f}{y}\right),
    \qquad \prt{f}{\bar{z}}
    = \frac{1}{2}\left(\prt{f}{x} + i\prt{f}{y}\right)
    \implies df = \prt{f}{z}dz + \prt{f}{\bar{z}}d\bar{z}
\end{align}
\begin{definition}[Harmonic]
We say a real or complex valued function \(f(x, y)\) is \textbf{harmonic} if \(f\) is \(\mc{C}^2\) and
\begin{equation}
  \prt{^2f}{x^2} + \prt{^2f}{y^2} \iff \prt{^2f}{z\partial\bar{z}} = 0
\end{equation}
\end{definition}
\begin{proposition}
  Every real-valued harmonic function is, not necessarily everywhere but at least \textit{locally}, the real part of a holomorphic function.
\end{proposition}

\begin{theorem}
  \(\omega\) has a primitive in \(\Omega\) if and only if, for any piecewise differentiable closed curve \(\gamma: [a, b] \to \Omega\) (i.e. with \(\gamma(a) = \gamma(b)\)), or equivalently any piecewise differentiable \(\gamma: S^1 \to \Omega\), we have
  \begin{equation}
    \int_\gamma\omega = 0
  \end{equation}
\end{theorem}

\begin{definition}
  We say a differential form \(\omega\) on a domain \(\Omega\) is \textbf{closed} if every point in \(\Omega\) has a neighborhood in which \(\omega\) has a primitive.
\end{definition}

\begin{theorem}
  Any closed differential form \(\omega\) in a simply-connected open set \(\Omega\) has a primitive.
\end{theorem}

\begin{theorem}[Cauchy's Theorem]
  Let \(\Omega\) be a domain and let \(f(z)\) be continuous in \(\Omega\) and holomorphic except on a set of discrete lines and points. Then the differentiable form \(f(z)dz\) is closed.
\end{theorem}

\begin{corollary}
  A holomorphic function \(f(z)\) locally has a primitive, which is holomorphic (i.e. a function \(F\) such that \(dF = f(z)dz\))
\end{corollary}

\begin{corollary}[Morera's Theorem]
  If \(f(z)\) is continuous in \(\Omega\) and \(df = f(z)dz\) is closed, then \(f(z)\) is holomorphic.
\end{corollary}

\section{Useful Tools}

\begin{itemize}

  \item Projection from the Riemann Sphere:
  \begin{equation}
    \pi: S^2 \setminus \{N\} \to \mbb{C}, \pi(x, y, t) = \frac{x + iy}{1 - t}
  \end{equation}

  \item Green's Formula:
  \begin{theorem}[Green's formula]
    \begin{equation}
      \int_\gamma Pdx + Qdy
      = \iint_A\left(\prt{Q}{x} - \prt{P}{y}\right)dxdy
    \end{equation}
  \end{theorem}

  \item Schwarz Reflection Principle:
  \begin{theorem}[Schwarz Reflection Principle]
    If \(f: H \to \mbb{C}\) is continuous on the closed upper half-plane \(H\), holomorphic on the open upper half-plane and takes real values on the real axis (i.e. \(f(\reals) \subseteq \reals\)) then it can be extended to an entire function by \(f(\ol{z}) = \ol{f(z)}\). More generally, this can be applied to reflecting any half-domain over any line.
  \end{theorem}

\end{itemize}

\section{Residues and Integrals}

\section{Elliptic Curves}

\end{document}
