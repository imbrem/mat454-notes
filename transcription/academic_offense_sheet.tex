\documentclass{article}
\usepackage[margin=1in]{geometry}
\usepackage{mat454}

\usepackage[backend=biber]{biblatex}
\addbibresource{references.bib}

\usepackage{hyperref}
\hypersetup{
  colorlinks,
  linkcolor={red!50!black},
  citecolor={blue!50!black},
  urlcolor={blue!80!black}
}

\title{MAT454 Academic Offense Sheet}
\author{Jad Elkhaleq Ghalayini}

\begin{document}

\maketitle

A quick collection of useful facts, theorems, and definitions for complex analysis. May be incorrect, and is certainly incomplete. Use at your own risk!

\tableofcontents

\newpage

\section{Basic Definitions and Theorems}
For \(f = u + iv\) holomorphic, we have
\begin{equation}
  2\prt{f}{\bar{z}} = \prt{f}{x} + i\prt{f}{y} = 0
  \iff \prt{u}{x} = \prt{v}{y} \land \prt{u}{y} = -\prt{v}{x}
\end{equation}
\begin{definition}
The \textbf{differential} of \(f\) is given by
\begin{equation}
  df = \prt{f}{x}dx + \prt{f}{y}dy = \prt{f}{z}dz + \prt{f}{\bar{z}}d\bar{z}
\end{equation}
\end{definition}
\begin{align}
  dz = dx + idy, \qquad d\bar{z} = dx - idy \iff
  dx = \frac{1}{2}(dz + d\bar{z}), \qquad dy = \frac{1}{2i}(dz - d\bar{z}) \\
  \prt{f}{z} = \frac{1}{2}\left(\prt{f}{x} - i\prt{f}{y}\right),
    \qquad \prt{f}{\bar{z}}
    = \frac{1}{2}\left(\prt{f}{x} + i\prt{f}{y}\right)
    \implies df = \prt{f}{z}dz + \prt{f}{\bar{z}}d\bar{z}
\end{align}
\begin{definition}[Harmonic]
We say a real or complex valued function \(f(x, y)\) is \textbf{harmonic} if \(f\) is \(\mc{C}^2\) and
\begin{equation}
  \prt{^2f}{x^2} + \prt{^2f}{y^2} \iff \prt{^2f}{z\partial\bar{z}} = 0
\end{equation}
\end{definition}
\begin{proposition}
  Every real-valued harmonic function is, not necessarily everywhere but at least \textit{locally}, the real part of a holomorphic function.
\end{proposition}

\begin{theorem}
  \(\omega\) has a primitive in \(\Omega\) if and only if, for any piecewise differentiable closed curve \(\gamma: [a, b] \to \Omega\) (i.e. with \(\gamma(a) = \gamma(b)\)), or equivalently any piecewise differentiable \(\gamma: S^1 \to \Omega\), we have
  \begin{equation}
    \int_\gamma\omega = 0
  \end{equation}
\end{theorem}

\begin{definition}
  We say a differential form \(\omega\) on a domain \(\Omega\) is \textbf{closed} if every point in \(\Omega\) has a neighborhood in which \(\omega\) has a primitive.
\end{definition}

\begin{theorem}
  Any closed differential form \(\omega\) in a simply-connected open set \(\Omega\) has a primitive.
\end{theorem}

\begin{theorem}[Cauchy's Theorem]
  Let \(\Omega\) be a domain and let \(f(z)\) be continuous in \(\Omega\) and holomorphic except on a set of discrete lines and points. Then the differentiable form \(f(z)dz\) is closed.
\end{theorem}

\begin{corollary}
  A holomorphic function \(f(z)\) locally has a primitive, which is holomorphic (i.e. a function \(F\) such that \(dF = f(z)dz\))
\end{corollary}

\begin{corollary}[Morera's Theorem]
  If \(f(z)\) is continuous in \(\Omega\) and \(df = f(z)dz\) is closed, then \(f(z)\) is holomorphic.
\end{corollary}

\begin{definition}
  Let \(\gamma: S^1 \to \Omega\) be a closed curve and \(a \notin \gamma(S^1)\) be a point not in the image of \(\gamma\). Then the \textbf{winding number of \(\gamma\) with respect to \(a\)} is given by the integral
  \begin{equation}
    w(\gamma, a) = \frac{1}{2\pi i}\int_\gamma\frac{dz}{z - a}
  \end{equation}
  This integral is an integer as it is the difference between two branches of \(\log\).
\end{definition}

\begin{theorem}[Cauchy's Integral Formula]
  If \(f(z)\) is holomorphic in \(\Omega\), \(a \in \Omega\) and \(\gamma: S^1 \to \Omega\) is a closed curve with \(a \notin \gamma\), then
  \begin{equation}
    \frac{1}{2\pi i}\int_\gamma\frac{f(z)dz}{z - a} = w(\gamma, a)f(a)
  \end{equation}
\end{theorem}

\begin{theorem}[Liouville's Theorem]
  A bounded holomorphic function on all of \(\mbb{C}\) is a constant.
\end{theorem}

\begin{definition}[Zero]
If \(f\) is holomorphic in a neighborhood of \(z_0 \in \mbb{C}\) and \(f(z_0) = 0\), we can write, for some \(k \in \mbb{N}\),
\begin{equation}
  f(z) = (z - z_0)^kf_1(z)
\end{equation}
where\(f_1(z)\) is nonvanishing near \(z_0\). In this case \(k\) is called the \textbf{order} or \textbf{multiplicity} of the \textbf{zero} \(z_0\)
\end{definition}

\begin{definition}[Meromorphic]
  A function \(f\) is \textbf{meromorphic} on an open \(\Omega \subseteq \mbb{C}\) if it is defined and holomorphic in the complement of a discrete set such that in some neighborhood of every point of \(\Omega\) we can write
  \(f(z) = g(z)/h(z)\)
  where \(g, h\) are holomorphic and \(h\) is not identically zero.
\end{definition}

\begin{definition}[Laurent expansion]
Holomorphic functions in an annulus \(r < |z| < R\) have a convergent \textbf{Laurent expansion} in an annulus
\begin{equation}
  \sum_{n = -\infty}^\infty a_nz^n = P(z) + R(z), \qquad P(z) = \sum_{n < 0}a_nz^n, \qquad R(z) = \sum_{n \geq 0}a_nz^n
\end{equation}
\end{definition}

\begin{theorem}
  Every meromorphic function \(f\) on \(S^2\) is rational.
  \label{theorem:meromorphic_rational}
\end{theorem}

\begin{definition}[Isolated singularity]
A holomorphic function in a \underline{punctured} disk \(0 < |z| < R\) has an \textbf{isolated singularity} at \(0\) if \(f(z)\) cannot be extended to be holomorphic at \(0\).
\end{definition}

\begin{theorem}[Weierstrass Theorem]
If \(0\) is an essential singularity, then for all \(\epsilon > 0\), \(f(\{0 \leq |z| \leq \epsilon\})\) is dense in \(\mbb{C}\).
\end{theorem}

\section{Useful Tools}

\begin{itemize}

  \item Projection from the Riemann Sphere:
  \begin{equation}
    \pi: S^2 \setminus \{N\} \to \mbb{C}, \pi(x, y, t) = \frac{x + iy}{1 - t}
  \end{equation}

  \item Green's Formula:
  \begin{theorem}[Green's formula]
    \begin{equation}
      \int_\gamma Pdx + Qdy
      = \iint_A\left(\prt{Q}{x} - \prt{P}{y}\right)dxdy
    \end{equation}
  \end{theorem}

  \item Schwarz Reflection Principle:
  \begin{theorem}[Schwarz Reflection Principle]
    If \(f: H \to \mbb{C}\) is continuous on the closed upper half-plane \(H\), holomorphic on the open upper half-plane and takes real values on the real axis (i.e. \(f(\reals) \subseteq \reals\)) then it can be extended to an entire function by \(f(\ol{z}) = \ol{f(z)}\). More generally, this can be applied to reflecting any half-domain over any line.
  \end{theorem}

  \item Fourier coefficients and Cauchy inequalities:
  \begin{equation}
    f(z) = \frac{1}{2\pi i}\int_\gamma\frac{f(\zeta)d\zeta}{\zeta - z} \implies f^{(n)}(z) = \frac{n!}{2\pi i}\int_\gamma\frac{f(\zeta)d\zeta}{(\zeta - z)^{n + 1}}
  \end{equation}

  \begin{equation}
    f(re^{i\theta}) = \sum_{n = 0}^\infty a_nr^ne^{i\pi\theta}, \qquad a_nr^n = \frac{1}{2\pi}\int_0^{2\pi}e^{-in\theta}f(re^{i\theta})d\theta
  \end{equation}

  \begin{equation}
    M(r) = \sup_\theta|f(re^{i\theta})| \implies |a_n| \leq \frac{M(r)}{r^n}
  \end{equation}

  \item The Mean Value Property (MVP): \textit{harmonic} functions satisfy
  \begin{equation}
    f(\text{center of disk}) = \ \text{mean value on boundary}
  \end{equation}

  \item The Maximum Modulus Principle (MMP): if \(f\) is a continuous complex-valued function on an open \(\Omega \subseteq \mbb{C}\) with the MVP, then it satisfies the MMP, that is, if \(|f|\) has a local maximum at a point \(a\) of \(\Omega\), then \(f\) is constant in a neighborhood of \(a\).

  \item Schwarz's Lemma:
  \begin{theorem}[Schwarz's Lemma]
    Suppose \(f(z)\) is holomorphic in \(|z| < 1\), \(f(0) = 0\) and \(|f(z)| < 1\). Then
    \begin{enumerate}

      \item \(|f(z)| \leq |z|\) if \(|z| < 1\)

      \item If \(|f(z_0)| = |z_0|\) at some \(z_0 \neq 0\), then \(f(z) = \lambda z\) for some \(|\lambda| = 1\).

    \end{enumerate}
  \end{theorem}

\end{itemize}

\section{Residues and Integrals}

\begin{definition}[Residue]
Let \(f(z)\) be a holomorphic function in a punctured disc centered around \(a\), and let \(\gamma\) be a closed curve lying entirely in the punctured disc (in particular, never touching \(a\)) with winding number \(w(\gamma, a) = 1\). We define the \textbf{residue} of the differential form \(f(z)dz\) (or ``of \(f\)") at \(a\) to be
\begin{equation}
  \Res_a(f) = \frac{1}{2\pi i}\int_\gamma f(z)dz = a_{-1}
\end{equation}
where \(a_n\) are the coefficients in the Laurent expansion of \(f\) at \(a\). Note that this is independent of the choice of curve \(\gamma\).
\end{definition}

\begin{definition}[Residue at \(\infty\)]
  Writing \(z = \frac{1}{z'}\), we have in coordinates at \(\infty\)
  \begin{equation}
    f(z)dz = -\frac{1}{z'^2}f(1/z')dz' = g(z')dz'
  \end{equation}
  We define
  \begin{equation}
    \Res_\infty(f) = \Res_0(g) = -a_{-1}
  \end{equation}
  where \(a_n\) are the terms of the Laurent expansion in \(|z| > R\).
\end{definition}

\begin{theorem}[Residue Theorem]
  Let \(\Omega \subset S^2\) be open and let \(f(z)\) be holomorphic in \(\Omega\) except perhaps on a discrete set of isolated points. Let \(\Gamma\) be the oriented (piecewise \(\mc{C}^1\)) boundary of a compact set \(K \subset \Omega\) not containing any singularity (either essential singularities or poles). Then
  \begin{equation}
    \int_\Gamma f(z)dz = 2\pi i\sum_a\Res_a(f)
  \end{equation}
  where \(a\) ranges over the singularities contained in \(K\), perhaps including \(\infty\).
\end{theorem}

\section{Elliptic Curves}

\end{document}
