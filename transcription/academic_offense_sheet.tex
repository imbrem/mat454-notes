\documentclass{article}
\usepackage[margin=1in]{geometry}
\usepackage{mat454}

\usepackage[backend=biber]{biblatex}
\addbibresource{references.bib}

\usepackage{hyperref}
\hypersetup{
  colorlinks,
  linkcolor={red!50!black},
  citecolor={blue!50!black},
  urlcolor={blue!80!black}
}

\title{MAT454 Academic Offense Sheet}
\author{Jad Elkhaleq Ghalayini}

\begin{document}

\maketitle

A quick collection of useful facts, theorems, and definitions for complex analysis. May be incorrect, and is certainly incomplete. Use at your own risk!

\tableofcontents

\newpage

\section{Basic Definitions and Theorems}
For \(f = u + iv\) holomorphic, we have
\begin{equation}
  2\prt{f}{\bar{z}} = \prt{f}{x} + i\prt{f}{y} = 0
  \iff \prt{u}{x} = \prt{v}{y} \land \prt{u}{y} = -\prt{v}{x}
\end{equation}
\begin{definition}
The \textbf{differential} of \(f\) is given by
\begin{equation}
  df = \prt{f}{x}dx + \prt{f}{y}dy = \prt{f}{z}dz + \prt{f}{\bar{z}}d\bar{z}
\end{equation}
\end{definition}
\begin{align}
  dz = dx + idy, \qquad d\bar{z} = dx - idy \iff
  dx = \frac{1}{2}(dz + d\bar{z}), \qquad dy = \frac{1}{2i}(dz - d\bar{z}) \\
  \prt{f}{z} = \frac{1}{2}\left(\prt{f}{x} - i\prt{f}{y}\right),
    \qquad \prt{f}{\bar{z}}
    = \frac{1}{2}\left(\prt{f}{x} + i\prt{f}{y}\right)
    \implies df = \prt{f}{z}dz + \prt{f}{\bar{z}}d\bar{z}
\end{align}
\begin{definition}[Harmonic]
We say a real or complex valued function \(f(x, y)\) is \textbf{harmonic} if \(f\) is \(\mc{C}^2\) and
\begin{equation}
  \prt{^2f}{x^2} + \prt{^2f}{y^2} \iff \prt{^2f}{z\partial\bar{z}} = 0
\end{equation}
\end{definition}
\begin{proposition}
  Every real-valued harmonic function is, not necessarily everywhere but at least \textit{locally}, the real part of a holomorphic function.
\end{proposition}

\begin{theorem}
  \(\omega\) has a primitive in \(\Omega\) if and only if, for any piecewise differentiable closed curve \(\gamma: [a, b] \to \Omega\) (i.e. with \(\gamma(a) = \gamma(b)\)), or equivalently any piecewise differentiable \(\gamma: S^1 \to \Omega\), we have
  \begin{equation}
    \int_\gamma\omega = 0
  \end{equation}
\end{theorem}

\begin{definition}
  We say a differential form \(\omega\) on a domain \(\Omega\) is \textbf{closed} if every point in \(\Omega\) has a neighborhood in which \(\omega\) has a primitive.
\end{definition}

\begin{theorem}
  Any closed differential form \(\omega\) in a simply-connected open set \(\Omega\) has a primitive.
\end{theorem}

\begin{theorem}[Cauchy's Theorem]
  Let \(\Omega\) be a domain and let \(f(z)\) be continuous in \(\Omega\) and holomorphic except on a set of discrete lines and points. Then the differentiable form \(f(z)dz\) is closed.
\end{theorem}

\begin{corollary}
  A holomorphic function \(f(z)\) locally has a primitive, which is holomorphic (i.e. a function \(F\) such that \(dF = f(z)dz\))
\end{corollary}

\begin{corollary}[Morera's Theorem]
  If \(f(z)\) is continuous in \(\Omega\) and \(df = f(z)dz\) is closed, then \(f(z)\) is holomorphic.
\end{corollary}

\begin{definition}
  Let \(\gamma: S^1 \to \Omega\) be a closed curve and \(a \notin \gamma(S^1)\) be a point not in the image of \(\gamma\). Then the \textbf{winding number of \(\gamma\) with respect to \(a\)} is given by the integral
  \begin{equation}
    w(\gamma, a) = \frac{1}{2\pi i}\int_\gamma\frac{dz}{z - a}
  \end{equation}
  This integral is an integer as it is the difference between two branches of \(\log\).
\end{definition}

\begin{theorem}[Cauchy's Integral Formula]
  If \(f(z)\) is holomorphic in \(\Omega\), \(a \in \Omega\) and \(\gamma: S^1 \to \Omega\) is a closed curve with \(a \notin \gamma\), then
  \begin{equation}
    \frac{1}{2\pi i}\int_\gamma\frac{f(z)dz}{z - a} = w(\gamma, a)f(a)
  \end{equation}
\end{theorem}

\begin{theorem}[Liouville's Theorem]
  A bounded holomorphic function on all of \(\mbb{C}\) is a constant.
\end{theorem}

\begin{definition}[Zero]
If \(f\) is holomorphic in a neighborhood of \(z_0 \in \mbb{C}\) and \(f(z_0) = 0\), we can write, for some \(k \in \mbb{N}\),
\begin{equation}
  f(z) = (z - z_0)^kf_1(z)
\end{equation}
where\(f_1(z)\) is nonvanishing near \(z_0\). In this case \(k\) is called the \textbf{order} or \textbf{multiplicity} of the \textbf{zero} \(z_0\)
\end{definition}

\begin{definition}[Meromorphic]
  A function \(f\) is \textbf{meromorphic} on an open \(\Omega \subseteq \mbb{C}\) if it is defined and holomorphic in the complement of a discrete set such that in some neighborhood of every point of \(\Omega\) we can write
  \(f(z) = g(z)/h(z)\)
  where \(g, h\) are holomorphic and \(h\) is not identically zero.
\end{definition}

\begin{definition}[Laurent expansion]
Holomorphic functions in an annulus \(r < |z| < R\) have a convergent \textbf{Laurent expansion} in an annulus
\begin{equation}
  \sum_{n = -\infty}^\infty a_nz^n = P(z) + R(z), \qquad P(z) = \sum_{n < 0}a_nz^n, \qquad R(z) = \sum_{n \geq 0}a_nz^n
\end{equation}
\end{definition}

\begin{theorem}
  Every meromorphic function \(f\) on \(S^2\) is rational.
  \label{theorem:meromorphic_rational}
\end{theorem}

\begin{definition}[Isolated singularity]
A holomorphic function in a \underline{punctured} disk \(0 < |z| < R\) has an \textbf{isolated singularity} at \(0\) if \(f(z)\) cannot be extended to be holomorphic at \(0\).
\end{definition}

\begin{theorem}[Weierstrass Theorem]
If \(0\) is an essential singularity, then for all \(\epsilon > 0\), \(f(\{0 \leq |z| \leq \epsilon\})\) is dense in \(\mbb{C}\).
\end{theorem}

\begin{definition}
  Let \(\Omega \subset \mbb{C}\) be open. We define \(\mbb{C}(\Omega)\) to be the \textbf{ring of continuous, complex-valued functions on \(\Omega\)} and \(\mc{H}(\Omega)\) to be the \textbf{subring of holomorphic functions on \(\Omega\)}
\end{definition}

\begin{definition}[Uniform convergence on compact subsets]
  We say that a sequence of functions \(\{f_n\} \subset \mc{C}(\Omega)\) \textbf{converges uniformly on compact subsets} if for all compact subsets \(K \subset \Omega\), \(\{f_n | K\}\) converges uniformly, i.e.
  \begin{equation}
    \forall \ \text{compact} \ K \subset \Omega, \forall \epsilon > 0,
      \exists N \in \mbb{N}, \forall m, n \geq N, \forall z \in K,
        |f_m(x) - f_n(x)| < \epsilon
  \end{equation}
\end{definition}

\begin{theorem}[Weierstrass]
  \begin{enumerate}
    \item \(\mc{H}(\Omega)\) is a closed subspace of \(\mc{C}(\Omega)\), i.e. if \(\{f_n\} \subset \mc{H}(\Omega)\) converges uniformly to \(f\) on compact sets then \(f = \lim_{n \to \infty}f_n \in \mc{H}(\Omega)\) is holomorphic.
    \item The mapping \(\mc{H}(\Omega) \to \mc{H}(\Omega)\) \(f \mapsto f'\) is continous, i.e. if \(\{f_n\} \subset \mc{H}(\Omega)\) converges uniformly to \(f\) on compact sets then \(\{f_n'\}\) converges uniformly to \(f'\) on compact sets.
  \end{enumerate}
\end{theorem}

\begin{corollary}
  Let \(\{f_n\}\) be a series of holomorphic functions. If \(\{g_n = \sum_{k = 0}^nf_k\}\) converges uniformly on compact subsets of \(\Omega\), then the sum
  \begin{equation}
    f = \sum f_n
  \end{equation}
  is holomorphic on \(\Omega\) and the series can be differentiated term by term.
\end{corollary}
\begin{proposition}
  Let \(\Omega\) be a domain. If \(\{f_n\} \subset \mc{H}(\Omega)\) converges uniformly on compact sets and each \(f_n\) vanishes nowhere in \(\Omega\) then \(f = \lim_{n \to \infty}f_n\) is either never zero or identically zero.
\end{proposition}

\begin{corollary}
  Let \(\Omega\) be a domain. If \(\{f_n\} \subset \mc{H}(\Omega)\) converges uniformly on compact sets and each \(f_n\) is one-to-one, then \(\lim_{n \to \infty}f_n\) is either one-to-one or constant.
\end{corollary}

\begin{definition}
  We say that \(\sum_{n = 1}^\infty f_n\) \textbf{converges uniformly} (respectively \textbf{converges uniformly absolutely}) on \(X \subset \Omega\) if all but finitely many \(f_n\) have no pole in \(X\) and form a uniformly convergent (respectively uniformly absolutely convergent) series on \(X\).
\end{definition}

\begin{definition}
  Let \(X \subset \mbb{C}\) and \(\mc{S} \subset \mc{C}(X)\). We say that \(\mc{S}\) is \textbf{equicontinuous} at \(a \in X\) if
  \begin{equation}
    \forall \epsilon > 0, \exists \delta > 0, \forall z \in X, |z - a| < \delta
      \implies \forall f \in \mc{S}, |f(z) - f(a)| < \epsilon
  \end{equation}
  \(\mc{S}\) is \textbf{equicontinuous on \(X\)} if it is equicontinous at each \(a \in X\). It is \textbf{uniformly equicontinuous on \(X\)} if
  \begin{equation}
    \forall \epsilon > 0, \exists \delta > 0, \forall z, w \in X, |z - w| <\delta
      \implies \forall f \in \mc{S}, |f(z) - f(w)| < \epsilon
  \end{equation}
\end{definition}

\begin{theorem}[Arzela-Ascoli]
Let \(\Omega \subset \mbb{C}\) be a \textbf{domain}. Then \(\mc{S} \subset \mc{C}(\Omega)\) is normal if and only if
\begin{enumerate}

  \item \(\mc{S}\) is equicontinuous on \(\Omega\)

  \item There exists \(z_0 \in \Omega\) such that \(\{f(z_0) : f \in \mc{S}\) is a bounded subset of \(\mbb{C}\)

\end{enumerate}
\end{theorem}

\begin{definition}
\(\mc{S} \subset \mc{C}(\Omega)\) is \textbf{locally bounded} on \(\Omega\) if
\begin{equation}\forall z_0 \in \Omega, \exists \delta > 0, M < \infty, \forall z \in \Omega, f \in \mc{S}, |z - z_0| < \delta \implies |f(z)| \leq M\end{equation}
This is true if and only if \(\mc{S}\) is \textbf{uniformly bounded on compact subsets of \(\Omega\)}, i.e. for all \(K \subset \Omega\) compact,
\begin{equation}\exists M = M(K), \forall z \in K, \forall f \in \mc{S}, |f(z)| \leq M\end{equation}
\end{definition}
\begin{theorem}[Montel]
Let \(\mc{S} \subset \mc{H}(\Omega)\) where \(\Omega \subset \mbb{C}\) is a domain. Then the following are equivalent:
\begin{enumerate}

  \item \(\mc{S}\) is normal

  \item \(\mc{S}\) is locally bounded

  \item \(\mc{S}' = \{f' : f \in \mc{S}\}\) is locally bounded and there exists \(z_0 \in\Omega\) such that \(\{f(z_0) : f \in \mc{S}\}\) is bounded in \(\mbb{C}\).

\end{enumerate}
\end{theorem}

The Arzela-Ascoili theorem holds for families of continuous functions with values in a complete metric space, e.g. continuos functions with values in the Riemann sphere \(S^2\) (or the extended complex plane \(\mbb{C}^* = \mbb{C} \cup \{\infty\}\)) with the (induced) \textbf{chordal metric}
\begin{equation}d(z, w) = \frac{2|z - w|}{\sqrt{(1 + |z|)^2(1 + |w|)^2}}\end{equation}
(we note that the topology induced by \(\mbb{C}\) by the chordal metric is the usual Euclidean topology).
\begin{definition}[Normal in the chordal metric]
A family \(\mc{S}\) of continuous functions on \(\Omega\) is \textbf{normal in the chordal metric} if and only if it is equicontinuous in the chordal metric: condition (2) of the Arzela-Ascoli theorem is not needed because \(\mbb{C}^*\) or \(S^2\) is compact in this topology.
\end{definition}
We can use this definition to analyze, e.g., a family \(\mc{S}\) of meromorphic functions on \(\Omega \subset \mbb{C}\) (or \(\Omega \subset S^2\)), since these can be considered holomorphic functions with values in \(S^2\).
\begin{lemma}
Let \(\{f_n\}\) be a sequence of meromorphic functions which converges uniformly on compact subsets of the domain \(\Omega \subset \mbb{C}\) (on \(S^2\), in the chordal metric). Then the limit function is either meromorphic or identically \(\infty\).
\end{lemma}

\begin{definition}[Spherical derivative]
  If \(f\) is meromorphic on a domain \(\Omega \subset \mbb{C}\) (or \(S^2\)), we define the \textbf{spherical derivative of \(f\)} at \(z \in \Omega\) by
  \begin{equation}
    f^\sharp(z) = \lim_{w \to z}\frac{d(f(z), f(w))}{|z - w|}
  \end{equation}
  If \(z\) is not a pole, we have
  \begin{equation}
    f^\sharp(z)
    = \lim_{w \to z}
      \frac{2|f(z) - f(w)|}{|z - w|\sqrt{(1 + |f(z)|^2)(1 + |f(w)|^2)}}
    = \frac{2|f'(z)|}{1 + |f(z)|^2}
  \end{equation}
\end{definition}

We have that
  \begin{equation}\left(\frac{1}{f}\right)^\sharp = f^\sharp\end{equation}
  implying that \(f^\sharp(z)\) is finite and continuous at all \(z \in \Omega\), and greater than zero at \(z\) if and only if \(f\) is one-to-one near \(z\).
  \begin{theorem}[Marty's Theorem]
  Let \(\mc{S}\) be a family of meromorphic functions on a domain \(\Omega\). Then \(\mc{S}\) is normal in the chordal metric if and only if
  \begin{equation}\mc{S}^\sharp = \{f^\sharp : f \in \mc{S}\}\end{equation}
  is bounded.
\end{theorem}

\begin{theorem}[Riemann mapping theorem]
  Any simply connected open \(\Omega \subset \mbb{C}\) except \(\mbb{C}\) itself has a biholomorphic mapping onto the open unit disc \(D\)
\end{theorem}

\begin{theorem}[Picard's Little Theorem]
A non-constant entire function \(f\) omits at most a point, i.e. \(\#(\mbb{C}\setminus f(\mbb{C})) \leq 1\)
\end{theorem}

\begin{theorem}[Picard's Big Theorem]
  If \(z_0\) is an isolated essential singularity of a holomorphic function \(f(z)\), then \(f\) takes every complex value with one possible exception in any neighborhood \(\Omega\) of \(z_0\), i.e. \(\#(\mbb{C}\setminus f(\Omega)) \leq 1\).
  \label{theorem:big_picard}
\end{theorem} 

\section{Useful Results and Formulas}

\begin{itemize}

  \item Projection from the Riemann Sphere:
  \begin{equation}
    \pi: S^2 \setminus \{N\} \to \mbb{C}, \pi(x, y, t) = \frac{x + iy}{1 - t}
  \end{equation}

  \item Green's Formula:
  \begin{theorem}[Green's formula]
    \begin{equation}
      \int_\gamma Pdx + Qdy
      = \iint_A\left(\prt{Q}{x} - \prt{P}{y}\right)dxdy
    \end{equation}
  \end{theorem}

  \item Schwarz Reflection Principle:
  \begin{theorem}[Schwarz Reflection Principle]
    If \(f: H \to \mbb{C}\) is continuous on the closed upper half-plane \(H\), holomorphic on the open upper half-plane and takes real values on the real axis (i.e. \(f(\reals) \subseteq \reals\)) then it can be extended to an entire function by \(f(\ol{z}) = \ol{f(z)}\). More generally, this can be applied to reflecting any half-domain over any line.
  \end{theorem}

  \item Fourier coefficients and Cauchy inequalities:
  \begin{equation}
    f(z) = \frac{1}{2\pi i}\int_\gamma\frac{f(\zeta)d\zeta}{\zeta - z} \implies f^{(n)}(z) = \frac{n!}{2\pi i}\int_\gamma\frac{f(\zeta)d\zeta}{(\zeta - z)^{n + 1}}
  \end{equation}

  \begin{equation}
    f(re^{i\theta}) = \sum_{n = 0}^\infty a_nr^ne^{i\pi\theta}, \qquad a_nr^n = \frac{1}{2\pi}\int_0^{2\pi}e^{-in\theta}f(re^{i\theta})d\theta
  \end{equation}

  \begin{equation}
    M(r) = \sup_\theta|f(re^{i\theta})| \implies |a_n| \leq \frac{M(r)}{r^n}
  \end{equation}

  \item The Mean Value Property (MVP): \textit{harmonic} functions satisfy
  \begin{equation}
    f(\text{center of disk}) = \ \text{mean value on boundary}
  \end{equation}

  \item The Maximum Modulus Principle (MMP): if \(f\) is a continuous complex-valued function on an open \(\Omega \subseteq \mbb{C}\) with the MVP, then it satisfies the MMP, that is, if \(|f|\) has a local maximum at a point \(a\) of \(\Omega\), then \(f\) is constant in a neighborhood of \(a\).

  \item Schwarz's Lemma:
  \begin{theorem}[Schwarz's Lemma]
    Suppose \(f(z)\) is holomorphic in \(|z| < 1\), \(f(0) = 0\) and \(|f(z)| < 1\). Then
    \begin{enumerate}

      \item \(|f(z)| \leq |z|\) if \(|z| < 1\)

      \item If \(|f(z_0)| = |z_0|\) at some \(z_0 \neq 0\), then \(f(z) = \lambda z\) for some \(|\lambda| = 1\).

    \end{enumerate}
  \end{theorem}

  \item Automorphisms of the complex plane:
  \begin{equation}
    \Aut\mbb{C} = \{\text{linear transformations} \ w = az + b, \quad a \neq 0\}
  \end{equation}

  \item Automorphisms of the Riemann Sphere:
  \begin{equation}
    w = \frac{az + b}{cz + d}, \quad ad - bc \neq 0
  \end{equation}
  Inverse:
  \begin{equation}
    \frac{dz - b}{-cz + a}
  \end{equation}

  \item Automorphisms of the upper half-plane:
  \begin{equation}
    w = \frac{az + b}{cz + d}, \quad a, b, c, d \in \mbb{R}
  \end{equation}

\end{itemize}

\section{Residues and Integrals}

\begin{definition}[Residue]
Let \(f(z)\) be a holomorphic function in a punctured disc centered around \(a\), and let \(\gamma\) be a closed curve lying entirely in the punctured disc (in particular, never touching \(a\)) with winding number \(w(\gamma, a) = 1\). We define the \textbf{residue} of the differential form \(f(z)dz\) (or ``of \(f\)") at \(a\) to be
\begin{equation}
  \Res_a(f) = \frac{1}{2\pi i}\int_\gamma f(z)dz = a_{-1}
\end{equation}
where \(a_n\) are the coefficients in the Laurent expansion of \(f\) at \(a\). Note that this is independent of the choice of curve \(\gamma\).
\end{definition}

\begin{definition}[Residue at \(\infty\)]
  Writing \(z = \frac{1}{z'}\), we have in coordinates at \(\infty\)
  \begin{equation}
    f(z)dz = -\frac{1}{z'^2}f(1/z')dz' = g(z')dz'
  \end{equation}
  We define
  \begin{equation}
    \Res_\infty(f) = \Res_0(g) = -a_{-1}
  \end{equation}
  where \(a_n\) are the terms of the Laurent expansion in \(|z| > R\).
\end{definition}

\begin{theorem}[Residue Theorem]
  Let \(\Omega \subset S^2\) be open and let \(f(z)\) be holomorphic in \(\Omega\) except perhaps on a discrete set of isolated points. Let \(\Gamma\) be the oriented (piecewise \(\mc{C}^1\)) boundary of a compact set \(K \subset \Omega\) not containing any singularity (either essential singularities or poles). Then
  \begin{equation}
    \int_\Gamma f(z)dz = 2\pi i\sum_a\Res_a(f)
  \end{equation}
  where \(a\) ranges over the singularities contained in \(K\), perhaps including \(\infty\).
\end{theorem}

\section{Elliptic Curves}

\begin{definition}
  Let \(e_1, e_2 \in \mbb{C}\) be linearly independent over \(\mbb{R}\). We can define a discrete subgroup of \(\mbb{C}\) with \textbf{basis} \(e_1, e_2\)
  \begin{equation}
    \Gamma = \{n_1e_1 + n_2e_2 : n_1, n_2 \in \mbb{Z}\}
  \end{equation}
  We say that \(f\) has \(\Gamma\) as \textbf{group of periods} if
  \begin{equation}
    \forall z \in \mbb{C}, f(z) = f(z + e_1) = f(z + e_2)
  \end{equation}
\end{definition}
\begin{definition}[Weierstrass \(\wp\)-function]
  We define the Weierstrass \(\wp\)-function by the infinite sum
  \begin{equation}\wp(z) = \frac{1}{z^2} + \sum_{\substack{w \in \Gamma \\ w \neq 0}}\left[
    \frac{1}{(z - w)^2} - \frac{1}{w^2}
  \right]\end{equation}
\end{definition}
\begin{claim}
  \begin{enumerate}
    \item \(\wp\) has a double pole at each \(w \in \Gamma\) with prime part \(\frac{1}{(z - w)^2}\)
    \item \(\wp\) is an even function
    \item \(\wp' = -z\sum_{w \in \Gamma}(z - w)^{-2}\) converges absolutely uniformly on compact subsets of \(\mbb{C}\)
    \item \(\wp'\) is doubly periodic: \(\forall w \in \Gamma, \wp'(z + w) = \wp'(z)\)
    \item \(\wp'\) is odd
    \item \(\wp\) itself has \(\Gamma\) as group of periods
  \end{enumerate}
\end{claim}
We have
\begin{equation}
  (\wp')^2 - 4\wp^3 + 20a_2\wp + 28a_4
\end{equation}

\begin{proposition}
  If \(f\) is a non-constant meromorphic function on \(\mbb{C}\) with \(\Gamma\) as group of periods, then the number of zeros of \(f\) in a period parallelogram is equal to the number of poles in the same parallelogram
\end{proposition}

\end{document}
