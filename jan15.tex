\documentclass{article}
\usepackage[utf8]{inputenc}

\title{MAT454 Notes}
\author{Jad Elkhaleq Ghalayini}
\date{January 15 2020}

\usepackage{amsmath}
\usepackage{amssymb}
\usepackage{amsthm}
\usepackage{mathtools}
\usepackage{enumitem}
\usepackage{graphicx}
\usepackage{cancel}
\usepackage{mathabx}

\usepackage[margin=1in]{geometry}

\newtheorem{theorem}{Theorem}
\newtheorem{lemma}{Lemma}
\newtheorem*{claim}{Claim}

\newcommand{\brac}[1]{\left(#1\right)}
\newcommand{\sbrac}[1]{\left[#1\right]}
\newcommand{\eval}[3]{\left.#3\right|_{#1}^{#2}}
\newcommand{\ip}[2]{\left\langle#1,#2\right\rangle}
\newcommand{\mb}[1]{\mathbf{#1}}
\newcommand{\mbb}[1]{\mathbb{#1}}
\newcommand{\mc}[1]{\mathcal{#1}}
\newcommand{\prt}[2]{{\frac{\partial {#1}}{\partial {#2}}}}
\def\ries{{\hat{\mbb{C}}}}
\newcommand{\reals}{\mbb{R}}

\newtheorem{definition}{Definition}
\newtheorem{proposition}{Proposition}

\DeclareMathOperator{\Res}{Res}

\begin{document}

\maketitle

Let's, as we usually do, consider a holomorphic function \(f(z)\) in an open set \(\Omega \subseteq \mbb{C}\). Last time, we showed that \(f\) has a convergent power series expansion in any open disc in \(\Omega\) (centered at the center of the disc). For example, around \(a = 0 \in \Omega\), we can write
\[f(z) = \sum_{n = 0}^\infty a_nz^n\]
Writing \(z = re^{i\theta}\), we get
\[f(re^{i\theta}) = \sum_{n = 0}^\infty a_nr^ne^{in\theta}\]
We can write out the following formula for these Fourier coefficients:
\[a_nr^n = \frac{1}{2\pi}\int_0^{2\pi}e^{-in\theta}f(re^{i\theta})d\theta\]
Today we're going to be looking at the consequences of this formula. First of all, this formula gives a simple but useful upper bound on \(a_n\): if we take the maximum absolute value of \(f\) along the circle of radius \(r\), written
\[M(r) = \sup_\theta|f(re^{i\theta})|\]
we get
\[|a_n| \leq \frac{M(r)}{r^n}\]
These are called \textbf{Cauchy's inequalities}. These have some important consequences, like \textbf{Liouville's theorem}: a bounded holomorphic function on all of \(\mbb{C}\) is a constant. How does this follow? Well, if \(c\) is the upper bound of \(f\) on \(\mbb{C}\), we have each
\[\forall r \in \reals^+, M(r) \leq c \implies |a_n| \leq \frac{M(r)}{r^n} \leq \frac{c}{r^n}\]
Hence, for \(n \leq 0\), \(0 \leq c \leq \epsilon\) for all \(\epsilon > 0\), implying \(c = 0\).

\end{document}
