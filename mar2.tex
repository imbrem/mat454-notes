\documentclass{article}
\usepackage[utf8]{inputenc}

\title{MAT454 Notes}
\author{Jad Elkhaleq Ghalayini}
\date{March 2 2020}

\usepackage{amsmath}
\usepackage{amssymb}
\usepackage{amsthm}
\usepackage{mathtools}
\usepackage{enumitem}
\usepackage{graphicx}
\usepackage{cancel}
\usepackage{mathabx}

\usepackage[margin=1in]{geometry}

\newtheorem{theorem}{Theorem}
\newtheorem{lemma}{Lemma}
\newtheorem*{claim}{Claim}

\newcommand{\brac}[1]{\left(#1\right)}
\newcommand{\sbrac}[1]{\left[#1\right]}
\newcommand{\eval}[3]{\left.#3\right|_{#1}^{#2}}
\newcommand{\ip}[2]{\left\langle#1,#2\right\rangle}
\newcommand{\mb}[1]{\mathbf{#1}}
\newcommand{\mbb}[1]{\mathbb{#1}}
\newcommand{\mc}[1]{\mathcal{#1}}
\newcommand{\prt}[2]{{\frac{\partial {#1}}{\partial {#2}}}}
\def\ries{{\hat{\mbb{C}}}}
\newcommand{\reals}{\mbb{R}}

\newtheorem{definition}{Definition}
\newtheorem{proposition}{Proposition}

\DeclareMathOperator{\Res}{Res}
\DeclareMathOperator{\BigP}{P}
\DeclareMathOperator{\Aut}{Aut}
\newcommand{\Prj}[2]{\BigP^{#1}({#2})}

\begin{document}

\maketitle

\section*{The Conformal Mapping Problem}

Let \(f\) be a holomorphism, and assume \(f'(z_0) = 0\). Then \(f^{-1}\) exists in a neighborhood of \(f(z)\), and \(f\) is \textbf{conformal} at \(z_0\) (preserves angles and their orientations). A nonconstant holomorphic mapping \(f: \Omega \to \mbb{C}\) is \textbf{open}, if it is one to one, then \(f\) is a homeomorphism onto its image \(f(\Omega)\), and \(f^{-1}\) is a holomorphism.
\begin{definition}
A \textbf{conformal} or \textbf{biholomorphic} mapping \(f: \Omega \to \Omega'\) is a holomorphic mapping with aholomorphic inverse.
\end{definition}
We are now faced with the \textbf{conformal mapping problem}:
\begin{itemize}

  \item Given domains \(\Omega, \Omega' \subset \mbb{C}\), are they biholomorphic?

  \item If so, can we find all biholomorphisms?

\end{itemize}
We note that, for \(f, g: \Omega \to \Omega'\), \(f, g\) are biholomorphisms if and only if \(g^{-1} \circ f \in \Aut\Omega\), the group of biholomorphisms of \(\Omega\) with itself. Furthermore, \(f\) induces a conjugation map
\[\Aut\Omega \to \Aut\Omega', \quad S \mapsto f \circ S \circ f^{-1}\]
Now let's consider some examples, starting with the complex plane itself. We have that
\[\Aut\mbb{C} = \{\text{linear transformations} \ w = az + b, \quad a \neq 0\}\]
Suppose \(w = f(z) \in \Aut\mbb{C}\). At \(\infty\), \(f\) has either an essential signularity or a pole. But we can show we don't have an essential singularity. On the other hand, what about when \(f\) is a polynomial, say of degree \(n\), then that must mean it is not one-to-one, because
\(f(z) = w\) has \(n\) distinct roots for almost every value of \(w\), except at roots of the derivative \(w = f(z), f'(z) = 0\). So \(n = 1\).

What about the Riemann sphere, \(\Aut S^2\). What should this group of biholomorphisms look like? Fractional linear transformations
\[w = \frac{az + b}{cz + d}, \quad ad - bc \neq 0\]
These coefficient, of course, are not uniquely determined, being only determined up to a constant. The inverse of a fractional linear transformation is that given by the inverse matrix, namely
\[\frac{dz - b}{-cz + a}\]
(uniquely determined up to a constant, so we don't have to write the \(\frac{1}{ad - bc}\)).

\end{document}
