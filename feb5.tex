\documentclass{article}
\usepackage[utf8]{inputenc}

\title{MAT454 Notes}
\author{Jad Elkhaleq Ghalayini}
\date{February 5 2020}

\usepackage{amsmath}
\usepackage{amssymb}
\usepackage{amsthm}
\usepackage{mathtools}
\usepackage{enumitem}
\usepackage{graphicx}
\usepackage{cancel}
\usepackage{mathabx}

\usepackage[margin=1in]{geometry}

\newtheorem{theorem}{Theorem}
\newtheorem{lemma}{Lemma}
\newtheorem*{claim}{Claim}

\newcommand{\brac}[1]{\left(#1\right)}
\newcommand{\sbrac}[1]{\left[#1\right]}
\newcommand{\eval}[3]{\left.#3\right|_{#1}^{#2}}
\newcommand{\ip}[2]{\left\langle#1,#2\right\rangle}
\newcommand{\mb}[1]{\mathbf{#1}}
\newcommand{\mbb}[1]{\mathbb{#1}}
\newcommand{\mc}[1]{\mathcal{#1}}
\newcommand{\prt}[2]{{\frac{\partial {#1}}{\partial {#2}}}}
\def\ries{{\hat{\mbb{C}}}}
\newcommand{\reals}{\mbb{R}}

\newtheorem{definition}{Definition}
\newtheorem{proposition}{Proposition}

\DeclareMathOperator{\Res}{Res}
\DeclareMathOperator{\BigP}{P}
\newcommand{\Prj}[2]{\BigP^{#1}({#2})}

\begin{document}

\maketitle

\begin{definition}[\(n\)-dimensional complex projective space]
We define
\[\Prj{n}{\mbb{C}} = \mbb{C}^{n + 1} \setminus \{0\} / \sim\]
where
\[(x_0,...,x_n) \sim (x_0',...,x_n') \iff \exists \lambda \in \mbb{C}, (x_0',...,x_n') = (\lambda x_0,..., \lambda x_n)\]
We denote the equivalence class of \((x_0,...,x_n)\) by \([x_0,...,x_n]\).
\end{definition}
\begin{definition}[Homogeneous coordinates]
We define coordinate charts \(U_i = \{[x_0, ..., x_n] \in \Prj{n}{\mbb{C}} : x_i \neq 0\}\) with affine coordinates \(U_i \to \mbb{C}^n\),
\[[x_0,...,x_n] \mapsto \left(\frac{x_0}{x_i},...,\frac{x_{i - 1}}{x_i}, \frac{x_{i + 1}}{x_i},...,\frac{x_n}{x_i}\right)\]
with inverse
\[(g_1,...,g_n) \mapsto [g_1,...,g_{i - 1}, 1, g_{i + 1},..,g_n]\]
\end{definition}
Using these coordinates, we have that \(\Prj{n}{\mbb{C}}\) has the structure of an \(n\)-dimensional complex manifold, as the transition mappings are rational. Let's take one of the charts here, say \(U_0\), to be \(\mbb{C}^n\). So
\[\Prj{n}{\mbb{C}} = U_0 \cup \ \text{everything else}\]
But what's everything else? So \(U_0\) is all the points where \(x_0 \neq 0\), so everything else is the set of points
\[\{x_0 = 0\} = \{[0, x_1,...,x_n]\} \simeq \Prj{n - 1}{\mbb{C}} \implies \Prj{n}{C} = U_0 \cup \Prj{n - 1}{\mbb{C}}\]
We call this copy of \(\Prj{n - 1}{\mbb{C}} \simeq \{x_0 = 0\}\) the \textbf{hyperplane at infinity}. This is like a generation of the Riemann sphere which we saw before, which we saw was given by \(S^2 = \Prj{1}{\mbb{C}}\). So when we talk about \(\Prj{2}{\mbb{C}}\), that's like having 2-complex coordinates with a line at infinity. Specifically, we can write it as
\[\Prj{2}{\mbb{C}} = \{[x, y, t]\} = \mbb{C}^2_{(x, y)} \cup \{t = 0\}\]
the \textbf{projective line at infinity}.
Now assume we have a curve \(X \subset \mbb{C}^2\) generated by the equation
\[y^2 = 4x^3 - 20a_2x - 28a_4\]
where the RHS has three distinct roots. We want to compute the \textbf{compactification of \(X\) in \(\Prj{2}{\mbb{C}}\)}. We can write this down in homogeneous coordinates
\[y^2t = 4x^3 - 20a^2xt^2 - 28a_4t^3\]
taking \(X'\) to be the solution set of this.
Why is this the right thing? When you look at \(\Prj{2}{\mbb{C}}\), and look in here at the set of points
\[\{[x, y, t]: t \neq 0\} \simeq \mbb{C}^2_{(x, y)}\]
we see that it is has homomorphism
\[[x, y, t] \mapsto \left(\frac{x}{t}, \frac{y}{t}\right)\]
Hence, we rewrite our eqaution in our new coordinates for \(\mbb{C}^2\),
\[\frac{y^2}{t^2} = 4\frac{x^3}{t^3} - 20a_2\frac{x}{t} - 28a_4\]
Now we can just multiply both sides by \(t^3\). So if you haven't seen this before, this takes a little bit of familiarity, but the actual operations involved are very simple operations. Of course, our \textit{original} \(X\) is a subspace of \(X'\). But how much have we added to \(X\)? Well, if we set \(t = 0\), we get \(x = 0\). So, how many points are we adding? One point, at \(\infty\):
\[X' = X \cup \{[0, 1, 0]\}\]
Now, in the neighborhood of any finite point, \(X'\) just looks like \(X\). What about in a neighborhood of the point at \(\infty\), \([0, 1, 0]\)? What does it look like?
So this point \([0, 1, 0]\) doesn't actually lie in the coordinate chart \(U_0 = \{x \neq 0\}\), it lies in \(U_1 = \{y \neq 0\}\). This chart has affine coordinates given by \((x', t') = (x/y, t/y)\). So what's the equation of \(X'\) in \textit{this} coordinate chart? It's
\[t' = 4x'^3 - 20a_2x't'^2 - 28a_4t'^3\]
In some neighborhood of \((x', t') = (0, 0)\) (the point at infinity), the implicit function theorem tells us that \(t'\) is a holomorphic function of \(x'\):
\[t' = 4x'^3 - 320a_2x'^7 + ...\]
As an extercise, we can take the Taylor series
\[t' = b_0 + b_1x + b_2x^2 + ...\]
plug it into the equation and solve successively for the coefficients. If you haven't done that before, do the exercise.

This tells us, in general, though, that \(t'\) is a function of \(x'\). So in a neighborhood of the point at infinity, \(X'\) looks like the graph of this function.

\end{document}
