\documentclass{article}
\usepackage[utf8]{inputenc}

\title{MAT454 Notes}
\author{Jad Elkhaleq Ghalayini}
\date{February 5 2020}

\usepackage{amsmath}
\usepackage{amssymb}
\usepackage{amsthm}
\usepackage{mathtools}
\usepackage{enumitem}
\usepackage{graphicx}
\usepackage{cancel}
\usepackage{mathabx}

\usepackage[margin=1in]{geometry}

\newtheorem{theorem}{Theorem}
\newtheorem{lemma}{Lemma}
\newtheorem*{claim}{Claim}

\newcommand{\brac}[1]{\left(#1\right)}
\newcommand{\sbrac}[1]{\left[#1\right]}
\newcommand{\eval}[3]{\left.#3\right|_{#1}^{#2}}
\newcommand{\ip}[2]{\left\langle#1,#2\right\rangle}
\newcommand{\mb}[1]{\mathbf{#1}}
\newcommand{\mbb}[1]{\mathbb{#1}}
\newcommand{\mc}[1]{\mathcal{#1}}
\newcommand{\prt}[2]{{\frac{\partial {#1}}{\partial {#2}}}}
\def\ries{{\hat{\mbb{C}}}}
\newcommand{\reals}{\mbb{R}}

\newtheorem{definition}{Definition}
\newtheorem{proposition}{Proposition}

\DeclareMathOperator{\Res}{Res}
\DeclareMathOperator{\BigP}{P}
\newcommand{\Prj}[2]{\BigP^{#1}({#2})}

\begin{document}

\maketitle

\begin{definition}[\(n\)-dimensional complex projective space]
We define
\[\Prj{n}{\mbb{C}} = \mbb{C}^{n + 1} \setminus \{0\} / \sim\]
where
\[(x_0,...,x_n) \sim (x_0',...,x_n') \iff \exists \lambda \in \mbb{C}, (x_0',...,x_n') = (\lambda x_0,..., \lambda x_n)\]
We denote the equivalence class of \((x_0,...,x_n)\) by \([x_0,...,x_n]\).
\end{definition}
\begin{definition}[Homogeneous coordinates]
We define coordinate charts \(U_i = \{[x_0, ..., x_n] \in \Prj{n}{\mbb{C}} : x_i \neq 0\}\) with affine coordinates \(U_i \to \mbb{C}^n\),
\[[x_0,...,x_n] \mapsto \left(\frac{x_0}{x_i},...,\frac{x_{i - 1}}{x_i}, \frac{x_{i + 1}}{x_i},...,\frac{x_n}{x_i}\right)\]
with inverse
\[(g_1,...,g_n) \mapsto [g_1,...,g_{i - 1}, 1, g_{i + 1},..,g_n]\]
\end{definition}
Using these coordinates, we have that \(\Prj{n}{\mbb{C}}\) has the structure of an \(n\)-dimensional complex manifold, as the transition mappings are rational.

\end{document}
