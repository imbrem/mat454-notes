\documentclass{article}
\usepackage[utf8]{inputenc}

\title{MAT454 Notes}
\author{Jad Elkhaleq Ghalayini}
\date{January 6 2020}

\usepackage{amsmath}
\usepackage{amssymb}
\usepackage{amsthm}
\usepackage{mathtools}
\usepackage{enumitem}
\usepackage{graphicx}
\usepackage{cancel}
\usepackage{mathabx}

\usepackage[margin=1in]{geometry}

\newtheorem{theorem}{Theorem}
\newtheorem{lemma}{Lemma}
\newtheorem*{claim}{Claim}

\newcommand{\brac}[1]{\left(#1\right)}
\newcommand{\sbrac}[1]{\left[#1\right]}
\newcommand{\eval}[3]{\left.#3\right|_{#1}^{#2}}
\newcommand{\ip}[2]{\left\langle#1,#2\right\rangle}
\newcommand{\mb}[1]{\mathbf{#1}}
\newcommand{\mbb}[1]{\mathbb{#1}}
\newcommand{\prt}[2]{{\frac{\partial {#1}}{\partial {#2}}}}
\def\ries{{\hat{\mbb{C}}}}
\newcommand{\reals}{\mbb{R}}

\newtheorem{definition}{Definition}
\newtheorem{proposition}{Proposition}

\DeclareMathOperator{\Res}{Res}

\begin{document}

\maketitle

The main objects of study in this course are holomorphic functions.
\begin{definition}[Holomorphic function]
\(f(t)\) is called \textbf{holomorphic at \(z \in \mbb{C}\)} if
\[\lim_{h \to 0}\frac{f(z + h) - f(z)}{h}\]
exists, i.e. there is \(c \in \mbb{C}\) such that
\[f(z + h) = f(z) + c \cdot h + \varphi(h) \cdot h, \lim_{h \to 0}\varphi(h) = 0\]
\end{definition}
Now, from this perspective, this looks no different from the usual case of a differentiable function. But it is different, because the variables are complex, and hence we can write
\[z = x + iy, \qquad f(z) = u(z) + iv(z)\]
Hence, this function mapping \(z \mapsto f(z)\) is, from the real perspective, a function from
\(\reals^2 \to \reals^2\), taking
\[\begin{pmatrix} x \\ y \end{pmatrix} \mapsto \begin{pmatrix} u(x, y) \\ v(x, y) \end{pmatrix}\]
Naturally, in the above definition, we can also write \(a + ib\) and \(h = \xi + i\eta\). Hence the derivative \(h \mapsto c \cdot h\) can be written as
\[\begin{pmatrix} \xi \\ \eta \end{pmatrix} \mapsto
\begin{pmatrix} a & -b \\ b & a \end{pmatrix}
  \begin{pmatrix} \xi \\ \eta \end{pmatrix}
= \begin{pmatrix}\prt{f}{x} & \prt{f}{y}\end{pmatrix}
  \begin{pmatrix} \xi \\ \eta \end{pmatrix}\]
In other words, this says that
\[\prt{f}{x} + i\prt{f}{y} = 0
\iff \prt{u}{x} = \prt{v}{y} \land \prt{u}{y} = -\prt{v}{x}\]
These are what is called the Cauchy-Riemann equations. So the moral of the story is that holomorphic is \textit{not} the same as differentiable as a function of two real variables. It' the same as differentiable as a function of two real variables \textit{plus} satisfying the Cauchy-Riemann equations.

It's going to convenient throughout this course to think about derivatives in terms of differential forms. Let's suppose, to begin a bit more generally, that we're considering a complex-valued \textit{differentiable} (not necessarily holomorphic) function \(f(x, y)\).
\begin{definition}
The \textbf{differential} of \(f\) is given by
\[df = \prt{f}{x}dx + \prt{f}{x}dy\]
\end{definition}
But, we're thinking about \(x\) and \(y\) as parts of a complex number, with \(z = x + iy\) and \(\bar{z} = x - iy\). So we can solve for \(x\) and \(y\) in terms of \(z\) and \(\bar{z}\). We can also compute the differentials
\[dz = dx + idy, \qquad d\bar{z} = dx - idy\]
So we can solve for \(dz\) and \(d\bar{z}\) in terms of \(dz\) and \(d\bar{z}\), getting
\[dx = \frac{1}{2}(dz + d\bar{z}), \qquad dy = \frac{1}{2i}(dz - d\bar{z})\]
In particular, we can take \(df\) and rewrite it in terms of \(dz\) and \(d\bar{z}\) by substituting in these expressions. So if we do that we get
\[df = \frac{1}{2}\left(\prt{f}{x} - i\prt{f}{y}\right)dz + \frac{1}{2}\left(\prt{f}{x} + i\prt{f}{y}\right)d\bar{z}\]
So, if we would like to define partial derivatives with respect to \(z\) and \(\bar{z}\), how should we define them? Well... the coefficients above seem to be natural choices, giving
\[\prt{f}{z} = \frac{1}{2}\left(\prt{f}{x} - i\prt{f}{y}\right),
  \qquad \prt{f}{\bar{z}}
= \frac{1}{2}\left(\prt{f}{x} + i\prt{f}{y}\right)
\implies df = \prt{f}{z}dz + \prt{f}{\bar{z}}d\bar{z}\]
In terms of \textit{this} expression, what's a third way of writing the Cauchy-Riemann equations? It's simply
\[\prt{f}{\bar{z}} = 0\]
And of course, this basically captures your ``feeling" of what a holomorphic function should be: it's supposed to be a function of \(z\), and not \(\bar{z}\). Ok, so this is the basic definition of holomorphic.

\end{document}
