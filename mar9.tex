\documentclass{article}
\usepackage[utf8]{inputenc}

\title{MAT454 Notes}
\author{Jad Elkhaleq Ghalayini}
\date{March 9 2020}

\usepackage{amsmath}
\usepackage{amssymb}
\usepackage{amsthm}
\usepackage{mathtools}
\usepackage{enumitem}
\usepackage{graphicx}
\usepackage{cancel}
\usepackage{mathabx}

\usepackage[margin=1in]{geometry}

\newtheorem{theorem}{Theorem}
\newtheorem{lemma}{Lemma}
\newtheorem*{claim}{Claim}

\newcommand{\brac}[1]{\left(#1\right)}
\newcommand{\sbrac}[1]{\left[#1\right]}
\newcommand{\eval}[3]{\left.#3\right|_{#1}^{#2}}
\newcommand{\ip}[2]{\left\langle#1,#2\right\rangle}
\newcommand{\mb}[1]{\mathbf{#1}}
\newcommand{\mbb}[1]{\mathbb{#1}}
\newcommand{\mc}[1]{\mathcal{#1}}
\newcommand{\prt}[2]{{\frac{\partial {#1}}{\partial {#2}}}}
\def\ries{{\hat{\mbb{C}}}}
\newcommand{\reals}{\mbb{R}}

\newtheorem{definition}{Definition}
\newtheorem{proposition}{Proposition}

\DeclareMathOperator{\Res}{Res}
\DeclareMathOperator{\BigP}{P}
\DeclareMathOperator{\Aut}{Aut}
\DeclareMathOperator{\Arg}{arg}
\DeclareMathOperator{\Id}{id}
\newcommand{\Prj}[2]{\BigP^{#1}({#2})}

\begin{document}

\maketitle

We're interested in a biholomorphism \(w = f(z)\) from a polygonal region enclosed by\(z_1,...,z_n\), with \(w_k = f(z_k)\) to the unit disc, where
\begin{itemize}

  \item \(0 < \alpha_k < 2\)

  \item \(-1 < \beta_k < 1, \sum\beta_k = 2\)

  \item \(\alpha_k + \beta_k = 1\),

  \item The intersection of the line between \(z_{k - 1}\) and \(z_k\) and the line between \(z_k\) and \(z_{k + 1}\) has angle \(\alpha_k\pi\) inside the polygon and \(\beta_k\pi\) outside the polygon

\end{itemize}
We want to find a formula for the inverse function \(z = F(w)\). The statement of the theorem (though last time we wrote it as an integral) is that, for some constant \(c\),
\[F'(w) = c\prod(w - w_k)^{\beta_k}\]
We have that \(\zeta = (z - z_k)^{1/\alpha_k}\) is invertible and maps the ``angle" \(\alpha_k\) to the half-disc. Writing
\[w = f(z_k + \zeta^{\alpha_k}) = g(\zeta), \zeta = (w - w_k)g(w) \implies F(w) = z_k + (w - w_k)^\alpha_kG_k(w)\]
\[\implies F'(w) = (w - w_k)^{\alpha_k - 1}G_k(w)\]
So
\[F'(w)(w - w_k)^{\beta_k}\]
is holomorphic and nonzero near \(w_h\). So
\[H(w) = F'(w)\prod(w - w_k)^\beta_k\]
is holomorphic and nonzero in a neighborhood of the closed unit disk. To show that \(H(w)\) is constant, it is enough to show that \(\arg H(w) = \Im\log H(w)\) is constant on \(S^1\) (this is well defined as zero is not included so there is a branch of log). This works because \(H\) is a harmonic function, and therefore we can use the Mean Value Property and the Maximum Modulus Principle.

So we just have to compute the argument. Let's look at what happens at a point \(e^{i\theta}\) on the arc between \(w_{k - 1}\) and \(w_k\). We compute
\[\frac{d}{d\theta}F(e^{i\theta}) = F'(e^{i\theta})ie^{i\theta}\]
We have that, since \(F(e^{i\theta})\) is a parametrization of a straight line,
\[\arg\frac{d}{d\theta}F(e^{i\theta}) = 0 \implies \arg F'(e^{i\theta}) = const - (\theta + \pi/2)\]
We have that
\[\arg(e^{i\theta} - w_k) = \theta/2 + const \implies \arg F'(e^{i\theta})\prod(e^{i\theta} - w_k)^{\beta_k} = const - \theta + (\sum\beta_k)\frac{\theta}{2} = const\]
This shows \(\arg H(w)\) is constant on the open arc from \(w_k\) to \(w_{k + 1}\) for all \(k\), but it's continuous because \(\log H(w)\) is well-defined. Therefore, \(H\) is constant on \(S^1\), completing the proof.


\end{document}
