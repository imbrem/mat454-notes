\documentclass{article}
\usepackage[utf8]{inputenc}

\title{MAT454 Notes}
\author{Jad Elkhaleq Ghalayini}
\date{March 4 2020}

\usepackage{amsmath}
\usepackage{amssymb}
\usepackage{amsthm}
\usepackage{mathtools}
\usepackage{enumitem}
\usepackage{graphicx}
\usepackage{cancel}
\usepackage{mathabx}

\usepackage[margin=1in]{geometry}

\newtheorem{theorem}{Theorem}
\newtheorem{lemma}{Lemma}
\newtheorem*{claim}{Claim}

\newcommand{\brac}[1]{\left(#1\right)}
\newcommand{\sbrac}[1]{\left[#1\right]}
\newcommand{\eval}[3]{\left.#3\right|_{#1}^{#2}}
\newcommand{\ip}[2]{\left\langle#1,#2\right\rangle}
\newcommand{\mb}[1]{\mathbf{#1}}
\newcommand{\mbb}[1]{\mathbb{#1}}
\newcommand{\mc}[1]{\mathcal{#1}}
\newcommand{\prt}[2]{{\frac{\partial {#1}}{\partial {#2}}}}
\def\ries{{\hat{\mbb{C}}}}
\newcommand{\reals}{\mbb{R}}

\newtheorem{definition}{Definition}
\newtheorem{proposition}{Proposition}

\DeclareMathOperator{\Res}{Res}
\DeclareMathOperator{\BigP}{P}
\DeclareMathOperator{\Aut}{Aut}
\DeclareMathOperator{\Arg}{arg}
\newcommand{\Prj}[2]{\BigP^{#1}({#2})}

\begin{document}

\maketitle

Today, our goal is to prove the Riemann mapping theorem:
\begin{theorem}[Riemann mapping theorem]
Any simply connected open \(\Omega \subset \mbb{C}\) except \(\mbb{C}\) itself has a biholomorphic mapping onto the open unit disc \(D\)
\end{theorem}
We will begin by proving a series of lemmas.
\begin{lemma}
There is a biholomorphism of \(\Omega\) onto a bounded open subset of \(\mbb{C}\)
\end{lemma}
\begin{proof}
Let \(a \notin \Omega\) be a point, which exists as \(\Omega \neq \mbb{C}\). Then \(\frac{1}{z - a}\) is a nonvanishing holomorphic function on the simply connected open set \(\Omega\), and so it has a primitive, some holomorphic function \(g(z)\).

Now, a primitive of \(\frac{1}{z - a}\) is like a branch of \(\log(z - a)\), which means that
\[z - a = e^{g(z)}\]
One thing this tells us right away is that \(g(z)\) is one to one, because \(z - a\) is one to one and if the composition of a function with something else is one to one, then that function must be one to one.

Take a point \(z_0 \in \Omega\). Since \(\Omega\) is open and \(g\) is one to one, implying it is nonconstant, there is an open disc centered at \(g(z_0)\) inside \(g(\Omega)\). Now, I claim that...
\end{proof}

\end{document}
