\documentclass{article}
\usepackage[utf8]{inputenc}

\title{MAT454 Notes}
\author{Jad Elkhaleq Ghalayini}
\date{January 17 2020}

\usepackage{amsmath}
\usepackage{amssymb}
\usepackage{amsthm}
\usepackage{mathtools}
\usepackage{enumitem}
\usepackage{graphicx}
\usepackage{cancel}
\usepackage{mathabx}

\usepackage[margin=1in]{geometry}

\newtheorem{theorem}{Theorem}
\newtheorem{lemma}{Lemma}
\newtheorem*{claim}{Claim}

\newcommand{\brac}[1]{\left(#1\right)}
\newcommand{\sbrac}[1]{\left[#1\right]}
\newcommand{\eval}[3]{\left.#3\right|_{#1}^{#2}}
\newcommand{\ip}[2]{\left\langle#1,#2\right\rangle}
\newcommand{\mb}[1]{\mathbf{#1}}
\newcommand{\mbb}[1]{\mathbb{#1}}
\newcommand{\mc}[1]{\mathcal{#1}}
\newcommand{\prt}[2]{{\frac{\partial {#1}}{\partial {#2}}}}
\def\ries{{\hat{\mbb{C}}}}
\newcommand{\reals}{\mbb{R}}

\newtheorem{definition}{Definition}
\newtheorem{proposition}{Proposition}

\DeclareMathOperator{\Res}{Res}

\begin{document}

\maketitle

\section*{Zeros and Poles}

\begin{definition}[Zero]
If \(f\) is holomorphic in a neighborhood of \(z_0 \in \mbb{C}\) and \(f(z_0) = 0\), we can write, for some \(k \in \mbb{N}\),
\[f(z) = (z - z_0)^kf_1(z)\]
where\(f_1(z)\) is nonvanishing near \(z_0\). In this case \(k\) is called the \textbf{order} or \textbf{multiplicity} of the \textbf{zero} \(z_0\)
\end{definition}
Zeros of holomorphic functions form a discrete set. We want to study, however, not only holomorphic functions, but also quotients of holomorphic functions
\begin{definition}[Meromorphic]
A function \(f\) is \textbf{meromorphic} on an open \(\Omega \subseteq \mbb{C}\) if it is defined and holomorphic in the complement of a discrete set such that in some neighborhood of every point of \(\Omega\) we can write
\(f(z) = g(z)/h(z)\)
where \(g, h\) are holomorphic and \(h\) is not identically zero.
\end{definition}
Why is it interesting to work with meromorphic and not just holomorphic functions? Essentially, it's because meromorphic functions in a domain \(\Omega\) form a field (whereas holomorphic functions only form a ring). Note that, in this course, when we say ``domain", what we mean is a connected open set.
If \(f(z), g(z)\) are holomorphic near \(z_0\), like before, we can write
\[f(z) = (z - z_0)^kf_1(z), \qquad g(z) = (z - z_0)^\ell g_1(z)\]
where \(f_1(z_0), g_1(z_0) \neq 0\). Near \(z_0\), then, the quotient looks like
\[\frac{f(z)}{g(z)} = (z - z_0)^{k - \ell}\frac{f_1(z)}{g_1(z)}\]
So what are the different possibilities? If \(k \geq \ell\), then this function extends to be holomorphic at \(z_0\). On the other hand, if \(k < \ell\), then, of course,
\[\lim_{z \to z_0}\left|\frac{f(z)}{g(z)}\right| = \infty\]
Note: \textit{not} undefined, but \(\infty\). In this case, we say that \(z_0\) is a pole of order \(\ell - k\).
Holomorphic functions in an annulus \(r < |z| < R\) have a convergent Laurent expansion in an annulus
\[\sum_{n = -\infty}^\infty a_nz^n = \sum_{n < 0}a_nz^n + \sum_{n \geq 0}a_nz^n\]
Note that the LHS converges when \(r < |z|\), whereas the RHS converges when \(|z| < R\). This actually comes from Cauchy's theorem, just in the case of a convergent power series expansion of a holomorphic function. So this is from Cauchy's integral formula:
\[f(z) = \frac{1}{2\pi i}\int_{\gamma_1}\frac{f(\xi)}{\xi - z}d\xi - \frac{1}{2\pi i}\int_{\gamma_2}\frac{f(\xi)}{\xi - z}d\xi = \sum_{n = -\infty}^\infty a_nz^n\]
for \(z\) between \(\gamma_1, \gamma_2\).
So
\[a_n = \frac{1}{2\pi i}\int_{\gamma_?}\frac{f(\xi)}{\xi^{n + 1}}d\xi\]
So of course, this should be just like before for the positive part. Note that this is the integral over \(\gamma_1\) if \(n \geq 0\) and over \(\gamma_2\) if \(n < 0\).

Previously, we talked about holomorphic functions on the Riemann sphere, noting there were very few of them: namely, constants. Now, there are a few more meromorphic functions on the Riemann sphere, but not much. Specifically,
\begin{theorem}
Every meromorphic function \(f\) on \(S^2\) is rational.
\end{theorem}
\begin{proof}
This theorem uses what is probably the only thing in first year calculus you don't prove: the partial fraction decomposition. So we prove it now.
Say \(f(z)\) has poles \(b_1,...,b_k\) (finite) and maybe \(\infty\). So what can we say about the Laurent expansion at a pole? There's only finitely many negative terms, specifically, the order of the pole.

So these negative parts of the Laurent expansions around each \(b_j\) are like polynomials \(P_j(\frac{1}{z - b_j})\). We'll call these principal parts. What about the principal part at \(\infty\)? It's a polynomial in \(z\), as it's a polynomial in \(\frac{1}{z'}\), where \(z'\) is the coordinate at \(\infty\), which is \(1/z\). Call this \(P_\infty(z)\). So we can write
\[f(z) - P_\infty(z) - \sum_{j = 1}^kP_j\left(\frac{1}{z - b_j}\right)\]
which is holomorphic on \(S^2\), and hence must be a constant \(a\). So we can write
\[f(z) = a + P_\infty(z) + \sum_{j = 1}^kP_j\left(\frac{1}{z - b_j}\right)\]
And that's rational.
\end{proof}

\end{document}
